\documentclass[11pt, oneside]{article}   	% use "amsart" instead of "article" for AMSLaTeX format
\usepackage{geometry}                		% See geometry.pdf to learn the layout options. There are lots.
\geometry{letterpaper}                   		% ... or a4paper or a5paper or ... 
%\geometry{landscape}                		% Activate for rotated page geometry
%\usepackage[parfill]{parskip}    		% Activate to begin paragraphs with an empty line rather than an indent
\usepackage{graphicx}				% Use pdf, png, jpg, or eps§ with pdflatex; use eps in DVI mode
								% TeX will automatically convert eps --> pdf in pdflatex		
\usepackage{amssymb}

%SetFonts

%SetFonts


\title{ProtoDUNE-SP TDR - Introduction}
\author{Flavio}
%\date{}							% Activate to display a given date or no date

\begin{document}
\maketitle

\section{ProtoDUNE-SP in context of DUNE/LBNF}

ProtoDUNE-SP is the single-phase DUNE Far Detector prototype that will be constructed and operated at the CERN Neutrino Platform (NP) starting in 2017. It was proposed to the CERN SPSC in June 2015 (SPSC-P-351), and following positive recommendations by SPSC and the CERN Research Board in December 2015, was approved at CERN as experiment NP-04 (ProtoDUNE). The Fermilab Director and the CERN Director of Research and Scientific Computing signed a Memorandum of Understanding (MoU) for this experiment in December 2015 that is valid until December 2022 in the first instance, and may be extended by mutual agreement. 

ProtoDUNE-SP is a significant experiment in its own. With its 0.77 kt of total LAr mass is the
largest monolithic single-phase LArTPC detector to be built so far. 
The CERN Neutrino Platform, extension to the EHN1 hall in the North Area, will provide a new dedicated charged-particle test beam line and ProtoDUNE SP aims at taking first beam data before the LHC LS2 (at the end of 2018). 

ProtoDUNE-SP is a crucial part the DUNE effort towards the construction of the first DUNE far detector "10-kt Module" (about 17 kt total LAr mass) prototyping the designs of most of the detector components at 1:1 scale and with an extrapolation of about 1:20 in total mass size, very similar to the scaling factor adopted in the past by ICARUS from the 10m$^3$ prototype to the T600 detector (two half-modules of about 375 t total LAr mass each).

The detector elements, the time projection chamber (TPC), the cold electronics (CE), and the Photon Detection System (PDS), are housed in a cryostat that contains the liquid argon target material. The cryostats is a free-standing steel-framed vessels with an insulated double-membrane system based on the technology used for liquefied natural gas (LNG) storage and transport ships. 
%It consists of an inner corrugated thin membrane of stainless steel surrounded by thermal insulation with a secondary thicker aluminum membrane placed between insulation layers. The structural steel frame provides the necessary support for the considerable hydrostatic pressure of the contained liquid.
A cryogenics system keeps the LAr at a stable temperature of about 89 K through a process of recovering evaporated argon, recondensing it and returning it to the cryostat via the closed loop 
for the forced recirculation of the liquid through the O$_2$ and H$_2$O filtration system that maintains the required LAr purity. 

The TPC includes the Anode Plane Assemblies (APA), the Cathode Plane Assemblies (CPA), and the field cage (FC). It consists of two arrays of three 6 m$\times$2.3 m wire-wrapped APAs at the opposite sides of the central CPA. Each APA is made of three parallel planes of wires (4.5 mm pitch, 2,560 wires) oriented at different angles, identical in design to those for the Far Detector. The CPA is held at $-$180 kV providing the 500 V/cm drift field in the 3.6 m drift regions between the CPA and APAs (also identical to the Far Detector configuration). Uniformity of the electric field is guaranteed by the FC delimiting the volume between CPA and APAs planes.

The CE, mounted onto the APA frame and immersed in LAr, amplify and continuously digitize the induced waveforms on the sense wires at several MHz, and transmit these data to the Data Acquisition system (DAQ), through the buffer to disk, then to the central CERN Tier-0 Computing Center and to other partner sites for analysis.  

The current PDS reference design uses a thin radiator coated with a wavelength-shifting layer of tetraphenyl-butadiene (TPB) that converts incoming VUV (128 nm) Ar scintillation photons to longer wavelength photons, in the visible blue range. The radiator is placed in front of a doped acrylic bar 200 cm long and 7.6 cm wide. Half of the converted photons will be emitted into the bar. A fraction of the wavelength-shifted optical photons are then internally reflected to the bar's end where they are detected by silicon photomultipliers (SiPM).
The APA frame is designed with ten bays into which PDS modules (coated bars) can be inserted. The bars are inserted into the frames after the TPC wires have been strung, allowing final assembly at the integration area at CERN NP prior to installation inside the cryostat. 

After completion of the detector mounting, LAr filling and commissioning of the prototype detector the charged-particle beam test will provide critical calibration measurements necessary for precise calorimetry as well as invaluable data sets to optimize the event reconstruction algorithms - interaction vertex finding and particle identification - and ultimately to quantify and reduce systematic uncertainties. These measurements are expected to significantly improve the physics reach of the DUNE experiment.

Validation of the membrane cryostat technology, associated cryogenics, and networking and computing infrastructure to handle the data and simulated data sets represent additional important goals. 

ProtoDUNE-SP will thus be an indispensable step to validate and benchmark the DUNE far detector engineering, to provide a well characterized charged-particle beam data set and to validate and test associated infrastructure requirements. 
Given its technical challenges, its importance to the DUNE experiment and the timeframe in which it must operate, ProtoDUNE-SP requires a strong organizational structure and a collaborative effort from most of the DUNE Collaboration components,  
from US National Laboratories and University groups, from CERN and from the international partners in EU and Latin America. 

In parallel, another ProtoDUNE (ProtoDUNE-DP) of identical size based on the dual-phase LArTPC technology will be built at CERN, as complementary part of the DUNE strategy for reaching the design goal of a 40 kton fiducial detector mass.
ProtoDUNE SP and ProtoDUNE DP share the CERN Neutrino Platform: this offers great opportunities for expanding common ``infrastructures" (e.g. cryo-system and LAr purification, cryo-instrumentation, on-line computing farm, off-line computing resources, ...) and for an optimal use and sharing of resources.

The ProtoDUNE SP detector is similar in design to SBND, the near detector of the SBN experiment at Fermilab. The timeline of the two experiments is also very similar: the development of effective synergies, the exploitation of common detector solutions, of common test tooling, and the optimal use of resources (human and financial) are all goals of on-going DUNE and SBN management effort. 

\subsection{ProtoDUNE-SP Management and Organization}
In the DUNE organizational chart, protoDUNE-SP is placed at the same high level as the Far Detector and the protoDUNE-DP organizations. This fosters a strong coupling with the Far Detector working groups and their activities as well as an efficient 
 overlap between the two protoDUNEs. It should be noted, however, that the protoDUNEs (SP and DP) are identified as distinct teams that take ownership of and responsibility for their separate experiments.
 
CERN and the Neutrino Platform Organization are fundamental partners for the 
 extension of the H2 (and H4) beam lines to ProtoDUNE SP (and ProtoDUNE DP) and for the delivery of the cryostat and the cryogenics infrastructure.

The ProtoDUNE SP leadership is the prime contact to the DUNE management as well as the interface with the CERN teams responsible for the Neutrino Platform and between CERN and the DUNE collaboration in all matters relating to ProtoDUNE SP.

ProtoDUNE SP is being organized in working groups dedicated to specific tasks. The construction of the TPC detector, of the Cold Electronics read-out boards and the Photon Detector System fall under responsibility of the respective Far Detector Working Groups. ProtoDUNE-SP {\bf TPC}, {\bf PDS} and {\bf Electronics} specifically appointed Teams take on the responsibility for the timely and technical success of component assembly, test and commissioning at CERN.

%These complex phase where multiple teams work in parallel and in compressed time schedule and space conditions require a strong technical coordination effort. A pool of experts with specific competence are appointed and stationary on the ground at CERN with specific coordination roles.

%The ProtoDUNE SP leadership is formed by two Coordinators and supported by a Deputy, and by a technical Lead.The Coordinators are prime contact to the DUNE management. The technical Lead liaises with the CERN teams responsible for the Neutrino Platform infrastructure for the installation and operation of ProtoDUNE SP.The ProtoDUNE SP leadership is the interface between CERN and the DUNE collaboration in all matters relating to ProtoDUNE SP.

%CERN and the Neutrino Platform Team are fundamental partners for the ProtoDUNE construction with the delivery of the membrane cryostat, the Cryostat Support Structure and the cryogenic and purification plant.

%ProtoDUNE SP is being organized in teams and working groups dedicated to specific tasks.TPC detector construction, Photon Detector construction and Read-out Electronics delivery both for TPC and PDS fall under responsibility of the respective Far Detector Working Groups. ProtoDUNE-SP {\bf TPC}, {\bf PDS} and {\bf Electronics} specifically appointed Teams take on the responsibility for the timely and technical success of detector assembly, test and commissioning at CERN.
%These complex phase where multiple teams work in parallel and in compressed time schedule and space conditions require a strong technical coordination effort. A pool of experts with specific competence are appointed and stationary on the ground at CERN with specific coordination roles.

A large ProtoDUNE-SP international team is dedicated to the {\bf DAQ system}. 
The other relevant detector components,  {\bf HV System},  {\bf Cryogenic Instrumentation} and {\bf Slow Control System}, are developed jointly by ProtoDUNE SP and ProtoDUNE DP teams, and all include significant contributions from CERN NP.

Teams dedicated to an external {\bf Cosmic Muon Tracker} (CRT) and to {\bf Beam Instrumentation} are responsible for design, construct, commission and operate corresponding hardware devices and for the DAQ interface.

A ProtoDUNE team at CERN is dedicated to {\bf Software Development} in connection with the LArSoft Collaboration at FNAL and support the off-line analysis activity of the {\bf Measurement} WG for a fast delivery of detector performance characterization and physics results based on beam and cosmics data.

At underlying level, {\bf Installation} and {\bf Engineering Integration} teams will provide the interface to CERN in all matters related to the infrastructures in the assembly hall and in the experimental hall, including the cryostat and cryogenic\&LAr purification systems of CERN Neutrino Platform responsibility.

\section{Goals of ProtoDUNE-SP}
The primary goal of the ProtoDUNE-SP test program at CERN is very clearly a twofold one: at one hand to benchmark and, if found fully adequate, to endorse the main technical solutions for the DUNE far detector components, and at the other one to perform the measurements needed to understand and possibly quantify the systematic uncertainties that will be present in the DUNE oscillation measurements. The latter goal anticipates physics outcomes relevant on their own.

The detector operation in real experimental conditions and for an extended run period will allow for a full characterization of all 
the components, from the membrane cryostat and the cooling and purification circuit, to the APA design and its read-out cold electronics layout and HV system, to the photon detection system and its read-out warm electronics.

The use of a well defined test beam of charged particles of known type and incident  energy, will significantly enhance the understanding the ultimate performance of the LArTPC technology and boost the optimization of event reconstruction, particle identification (PID) algorithms and calorimetric energy measurement.  The beam measurements will serve both as a calibration data set to tune the Monte Carlo simulations and as a reference data set for the DUNE experiment. 

Pion and proton beams from around one to few GeV will be used primarily to study hadronic interaction mechanisms, secondary particle production and, at higher energies, shower reconstruction and energy calibration. Electrons will be used to benchmark and tune electron/photon separation algorithms, to study electromagnetic cascade processes and to calibrate electromagnetic showers at higher energies. Charged kaons produced in the tertiary beam line are rare but are copiously produced by the pion beam interactions inside the detector. These will be extremely useful to characterize kaon identification efficiency for proton decay sensitivity.  Samples of stopping muons with Michel electrons from muon decay (or without, in case of negative muon capture) will be used for energy calibrations in the low energy range of the SN neutrino events and for the development of charge-sign determination methods. 

A cumulative test beam run period for ProtoDUNE-SP of four to eight weeks is assumed, depending on the extent of beamline sharing with other users at EHN1. This will take place prior to the long shutdown of the LHC in late 2018. 

During no-beam time cosmic runs will be acquired. A dedicated external trigger system consisting of arrays of scintillator paddles, suitably positioned and arranged in coincidence trigger logic, will select specific classes of cosmic muon events and enable the DAQ, e.g. long muon tracks crossing the entire detector at large zenith angles for an overall test of the detector performance and muons stopping inside the LAr volume for accurate Michel electron spectra accumulation and energy calibration in the low-energy range.

It is important to note that, besides calibration purposes and detector performance characterization, the unprecedented event reconstruction capability of the LArTPC technology combined with the large active volume of the ProtoDUNE-SP detector exposed to the CERN charged-particle beams open the way to a truly reach program of new physics investigations on particle interaction processes. 
  The LArTPC features simultaneously precise tracking (3D imaging detector) and accurate measurement of energy deposited (homogeneous calorimeter). The large size of the active volume in ProtoDUNE allows for good containment for the hadronic and electromagnetic interaction products in the few GeV range. No other detector ever had all these features combined in one. 
 
 Hadrons ($\pi$, K and $p$) interactions on Ar target around 1 GeV produce low multiplicity final states rather than "hadron showers", 
 and 1 GeV-electrons  (critical energy $\simeq 30$ MeV in Ar) produce low-populated cascades, with only few tens of secondary energetic electrons(positrons). 
"TPC/imaging-aided calorimetric measurements" in this energy range may allow to investigate 
energy deposition mechanisms and reconstruction methods where the usual hadronic and el.m. shower concepts and features are not well defined or cannot apply.
Calorimetric measurement of the energy deposited can be accomplished,  whenever possible, for each individual secondary particle/track thanks to the imaging capabilities of this type of detector.

In particular, the determination of the el.m. content in hadron cascades - $\pi^0$ multiplicity and the carried energy fraction as a function of primary hadron incident energy, never directly measured in Ar and only based on MC expectations - is per se a relevant scientific topic with impact in the understanding and modeling of the physics of intranuclear cascades in Ar nuclei.

Similarly, the measurement of the $e/\pi$ ratio and the study of compensation in a homogeneous LAr calorimeter with additional
 TPC/imaging features is of great interest for deepening the physics of particle detectors and for the modeling of Nuclear de-excitation and Nucleon evaporation processes. To this purpose,
 appropriate reconstruction algorithms with  LArTPC imaging-calorimeter down to very low energy thresholds for the detected TPC wire signal hit have to be defined or further developed, e.g. implementing new LArTPC dedicated 3D Energy Flow algorithms.
 
In conclusion, test beam data will be essential for a rich physics program with ProtoDune at the CERN Neutrino Platform.



\end{document}  


