
\section{Calibration Strategy}
%\fixme{from Josh K}
%%%%%%%%%%%%%%%%%%%%%%%%%%
%\subsection{Motivation}


	The scientific program of protoDUNE is critical to the ultimate success
of DUNE.  ProtoDUNE will be the DUNE Collaboration's only opportunity to
measure the response of a DUNE-style LAr-TPC to hadrons (as well as $\gamma$s,
$\mu$s, and electrons), and to compare those measurements to a model of the
detector response. That model can then be used to predict the response for DUNE
itself.  Precision measurements of response in testbeams are a common approach
of long-baseline neutrino experiments. As a recent example, NO{$\nu$}A has
found that some of their measurements are limited by the lack of knowledge of
this response, which a testbeam would have provided.
	
	By itself, however, protoDUNE does {\it not} provide a response
measurement that can simply be mapped onto DUNE.  Rather, protoDUNE provides
measurements of the part of the response that depends on the interactions of
various particle species in LAr.  The remainder of the detector response, which
includes reconstruction resolutions and biases, the effects of noise, space
charge and how it impacts reconstruction, and the conversion from observed
charge to energy, all must be calibrated and ultimately removed from the
physical hadron response.  Misinterpreting, for example, longitudinal diffusion
as a fundamental part of shower development of $\pi^+$s means our understanding
of $\pi$s will be wrong in DUNE---and there will be no test of that response in
the DUNE FD, independent of neutrino events themselves.  By building protoDUNE
in a way that mimics the DUNE far detector, one would still have to assume that such errors will
`cancel out';
 in practice one cannot rely on such a cancellation without knowing
how different the parameters governing the response may be.

	Ultimately, what this means is that one must calibrate a model of the
detector---likely a LArSoft model---and use it to predict the response to
hadrons.  If the detector model is accurate, differences between the predicted
response to various hadrons and measurements allows a correction to the hadron
response that can be used for the DUNE experiment.  

	The existing option for creating a calibrated detector model for
protoDUNE is to presume that calculations of the electric field throughout the
detector, including space charge effects and impurities, are accurate, and that
with this as input the detector response is uniquely calculable throughout the
volume and for all times, up to the details of the electronics transfer
function (and of course the response to various particle species which is what
is being measured by protoDUNE).  

	Figure~\ref{fig:35t} shows the variations in electron lifetime within
\fixme{where is figure in this doc ?}
the 35 tonne prototype as a function of position, as measured by {\it in-situ}
purity monitors. The clear differences top and bottom indicate that we cannot
assume that measurements in just a few places represent the detector as a
whole. In addition to these variations, changes with the ambient environment
were also seen, indicating that measurements must be made with reasonable
frequency as well.  An idealized view of how an operating detector will behave
would miss such variations.
	 
	        The calibration program for DUNE has as its goal both the
measurement of the parameters governing detector response and {\it tests} of
the predicted response.  Table~\ref{tbl:params} lists many of the critical
%
\begin{table}
\begin{center}
\begin{tabular}{|l| l| l| l| l| } \hline \hline
{\bf Parameter} & {\bf Name} & {\bf Ex-situ Calibration} & {\bf Calibration} & {\bf Test} \\ \hline
$W$ & Ionization Energy  & Benchtop & None & Cosmics \\
$C$ & ADC/Charge Map & Benchtop Pulsers & Front-End Pulsers & Stopped cosmics \\
$f$ & Energy Scale & Calculation & Stopped cosmics & Through-going cos. \\
$R$ & Electron recombination & Calculation & Through-going cosmics & Michels\\
$\tau$ & Electron lifetime & Purity Monitors & Through-going cosmics & Stopped
cosmics\\
$\vec{E}(\vec{r})$ & Electric Field map & Calculation & Crossing cosmics/halo &
Michels \\
$v_d$ & Drift Speed & Purity+temp+Calc.& triggered cosmics & None \\
$d$ & Electron Diffusion & Calculation & triggered cosmics & None \\
$\rho_E$ & Field response & Calculation & Through-going cosmics & None \\ \hline \hline
\end{tabular}
\caption{Parameters to be calibrated for ProtoDUNE model\label{tbl:params}}
\end{center}
\end{table}
%
parameters that need to be measured, and what will be available in protoDUNE
for both measurements of the parameters and tests of the resulting calibrated
model.  Unfortunately, time and budget constraints will prevent us from
building and deploying a laser system like that used in other surface
single-phase LAr-TPCs.  ProtoDUNE calibrations will rely solely on cosmic rays---those triggered by
going through a set of external tracking scintillation panels.  
%ProtoDUNE will
%be the first large-scale, surface single-phase LAr-TPC to make physics
%measurements using only calibrations from cosmic rays, and will therefore
%represent a triumph of the DUNE ``no LAr-TPC calibration device needed''
%philosophy.
%%% --- I like this sentence but it may be better not to trigger those kinds of discussions with the SPSC reviewers - TK ---

It is assumed that there will be
several auxiliary systems that already provide measurements or estimates of
some of these parameters:
\begin{itemize}
\item Monitors of HV and current to the APAs \fixme{verify whether this is the case}
\item Survey of wire positions \fixme{verify whether this is planned}
\item Temperature sensors {\it in situ} at several places in the volume
\item Purity monitors at several positions within the volume
\item Front-end electronics response calibration pulsers.
\end{itemize}
        These systems provide the initial input to the model for electric
field, and for calculations of average drift velocity, electron lifetime, and
electron diffusion.  The goal of the calibration system described here 
is first to measure the response parameters with granularity in time and
position that cannot be accomplished by the above systems, and to test the
resultant model with tracks of known positions, trajectories, and energy
deposits.


\fixme{xxxxxxxxxxxxxxxxxxxxxxxxxxxxxxxxxxxxxxxxxxx}
The planned cosmic muon tagging hardware is described in section \ref{sec:muontagger}

        Although there are many possible configurations for deployment of the
counters, a particularly interesting deployment would be fore-and-aft,
therefore triggering on muons that had trajectories similar to beam events. In
this configuration, the counters could also serve as a tagger for beam halo
events.  By tagging such events, only those whose energy and position are
reasonably well constrained will be part of the beam calibration.

        To fully measure the field map using cosmic rays, crossing
tracks to resolve ambiguities in the field direction are needed.  Our studies have shown
that even with counters just fore-and-aft, we can illuminate the detector
using halo muons that pass through both counters (i.e., roughly horizontally)
that cross cosmic tracks going through just one counter. In this case, the
trajectory of the cosmic ray inside the TPC is determined by the TPC itself.
While this argument appears problematic since it is the TPC we are attempting
to calibrate in the first place, we anticipate that the longitudinal field is
known well enough, even in the presence of large space charge effects, that
uncertainties in the drift velocity will be small.  While this has not yet been
quantitatively demonstrated in a real large-scale detector, for ProtoDUNE the plan 
is to use cosmics that go through both sets of counters to determine the drift
velocity field map (independently of determination of the electric field map)
and to use that map (rather than the electric field map) to reconstruct the
crossing tracks.  The collaboration has not yet explored whether the correlation between
these two approaches will lead to an unacceptably large uncertainty on the
field.

 While the fore-aft deployment scheme works well for calibration plans,
it provides little help on tagging cosmics that could cause difficulty for
reconstruction algorithms, since the vast majority of those will be
downward-going.  Reconstruction, however, has progressed significantly and
there is no expectation that existing algorithms will have problems identifying
beam-related tracks and showers beneath the rain of cosmics. There is therefore
no need for counters above and below the detector.

        In addition to using the triggered horizontal cosmics to measure the
drift velocity map, and the crossing halo muons with high-angle cosmics to
determine the transverse field components, we will also use stopped cosmics to
provide a measure of the detector's energy scale. Stopped cosmics have the
advantage that about a meter upstream of the track endpoint the muon is a MIP
and dE/dx is very well known. The techniques of using these has been
demonstrated very successfully in both MINOS and NOvA, and MicroBooNE has made
excellent progress in using these as a calibration as well. Once stopped cosmics have been 
used as a calibration source, the quality of the calibration can be tested by looking at the Michel
spectrum (using the same stopped events) and ultimately the mass peak of
reconstructed $\pi^0$s produced from charged pion beam events.


%%%%%%%%%%%%%%%%%%%%%%%%%%
%%\subsubsection{PDS Calibration Run Plan}
 %%    Run/Measurement plan  -- Toups ?
  %%   Light yield vs. field
   %%  timing studies,
   %%  beam off trigger

