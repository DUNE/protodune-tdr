%\chapter{sciencetech}

%%%%%%%%%%%%%%%%%%%%%%%%%%%%%%%%%%%%%%%%%%%%%%
\section{Charged particle beam studies}
Science description


%%%%%%%%%%%%%%%%%%%%%%%%%%%%%%%%%%%%%%%%%%%%%%
\section{Evaluation of event reconstruction performance}

Xin's input

In the following, we define the goals that we must achieve regarding the TPC signal processing.
\begin{itemize}
\item Number of unusable channels is required to be smaller than 1\%: \\
All the past large LArTPC experiments (ICARUS, MicroBooNE, and DUNE-35ton) suffer from a sizable
number unusable channels ($\sim$ 10\%) due to various reasons such as high noises, dead electronics.
It is crucial to demonstrate the number of unusable channels can be reduced by at least one order
of magnitude. 
\item The ENC generated beyond the cold preamplifier is required to negligible compared to the 
ENC generated by the cold preamplifier in > 95\% of the channels: \\
The only irreducible electronics noise is coming from the preamplifier on the TPC wires. It is thus
crucial to optimize the overall system to reduce the noises generating from other sub-systems. 
\item A robust procedure of the calibration program of the TPC field response function is required
be developed: \\
Field response on the induction plane is very complicated, but holds the key to achieve a robust TPC 
signal processing. It is therefore important to develop a program to calculate and validate field 
response functions in a real experiment. 
\item The overall signal calibration process is required to be validated: \\
The goal of the signal calibration process is to recover the number of the ionized electrons based on
the digitized TPC waveform, and is the first step in the overall event reconstruction. It is crucial to 
validate that such a procedure is robust.
\end{itemize}

%\subsection{TPC Signal Processing (Xin)}

%%%%%%%%%%%%%%%%%%%%%%%%%%%%%%%%%%%%%%%%%%%%%%
\section{Particle interactions and cross sections}

%%%%%%%%%%%%%%%%%%%%%%%%%%%%%%%%%%%%%%%%%%%%%%
\section{Detector engineering validation}
engineering motivation

%%%%%%%%%%%%%%%%%%%%%%%%%%%%%%%%%%%%%%%%%%%%%%
\section{Installation validation}




