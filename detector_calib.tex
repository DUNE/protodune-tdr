
\section{Detector-physics studies and calibrations}

Electromagnetic showers are relatively well-understood and can be used to understand the detector response.

	\subsection{ADC to energy calibration}
		\paragraph{Stopping/decaying muons}
		\paragraph{Michel electrons}

Neutrinos produced in supernova burst (SNB) explosions are calculated to be in the few-to-30-MeV range.
The DUNE Far Detector is expected to have a unique sensitivity to the $\nu_e$'s from an SNB in our galaxy.
%
The test beam cannot offer a sample of such low-energy electrons, but Michel electrons produced by decaying stopped muons are ideal for calibrating electron response in the appropriate 10-50~MeV energy range. 
The required  sample of low-energy muons will supply the 1400 Michel electrons needed for a 1\%  calibration of electrons in this energy range. This sample can be compared to a similar Michel electron sample
stemming from stopping cosmic muons.


	\subsection{Detector response uniformity}
		\paragraph{E-field response/space charge effects}
		
		A robust procedure to validate the TPC field response function is required be developed: \\
Field response on the induction plane is very complicated but holds the key to achieving a robust TPC  signal processing and thus the overall event reconstruction. It is, therefore, important to develop a 
program to calculate, calibrate, and validate field response functions. This validation of the field response function can be achieved by comparing the measured average response functions for cosmic muons with the simulated
ones. 

%%% \item [Measure field distortion effect (space-charge, LAr flow, beam window effect, etc).] 

The protoDUNE-SP detector will be placed on surface and effects coming from the space-charge modification of the electric field, or the effect of the choice of the location and design of the beam window will need to be included. The effects and strategies how to approach them are described in sections 3.5 and 7.1.
		
		\paragraph{APA alignments}

	\subsection{LAr purity monitoring}

	\subsection{Diffusion studies}

	\subsection{Noise studies}
  
  The ENC generated beyond the cold preamplifier is required to negligible ($<$ 50\%~\footnote{Since the noise from each source are added in quadrature, an extra electronic noise with less than 50\% of intrinsic cold preamplifier
noise would lead to less than 10\% increase of the total noise.}) compared to the 
expected ENC generated from the cold preamplifier in $>$ 95\% of the channels: \\
The only irreducible electronics noise is coming from the preamplifier on the TPC wires. It is thus
crucial to optimize the overall system to reduce the noises generating from other sub-systems. 
Lariat achieved such a performance with about a total of 500 channels. 


	\subsection{Recombination}
	 (angular dependency)


