%%%%%%%%%%%%%%%%%%%%%%%%%%%%%%%%%%%%%%%%%%%%%%
\section{Beam requirements}
\label{sec:beamrequirements}

The requested beam parameters are driven by the requirement that the results from the CERN test beam should be directly applicable to the future large underground single-phase LAr detector with minimal extrapolation. The CERN test beam data will be used to evaluate the detector performance, to understand the various physics systematic effects, and to provide ``neutrino-like'' data for event reconstruction studies. To satisfy the requirement, the beam parameters must span a broad range of particle spectrum that are expected in the future neutrino experiment. The particle beam composition should consist of electrons, muons, and hadron beams that are charge-selected. The expected momentum distributions for secondary particles from neutrino interactions are shown in Figure~\ref{fig:ParticleMomenta}. There is a large spread in the momentum distribution with most particles peaked near 200 MeV/c.  The desirable momentum range for ProtoDUNE-SP  is in the low momentum region. Based on the feedback and constraints from the CERN beam group, the beamline will be designed to allow the transport of beam particles from about 0.2 GeV/c up to 7 GeV/c. 

The maximum electron drift time in the TPC is about 2.2 ms. In order to minimize pile-up in the TPC, the planned beam rate should be around 200 Hz.  The ProtoDUNE-SP TPC has two drift volumes separated by a passive cathode plane.
It is desirable to aim the particle beam such that a large fraction of the lower energy hadronic showers are mostly contained in one drift 
volume to minimize uncertainties due to the passing of inactive detector materials.

The summary of the beam requirements are shown in Table~\ref{tab:beamspecs}.

\begin{cdrtable}[Particle beam requirement]{cc}{beamspecs}{Particle beam requirements}
%\textbf{Parameter } & \textbf{Requirements}  \\ \hline
 Parameter & Requirements \\ \toprowrule
  Particle Types        & $e^\pm,\mu^\pm,\pi^\pm$,$K$,$p$  \\ \colhline
  Momentum Range   & 0.2 - 7 GeV/$c$ \\ \colhline
  Momentum Spread   & $\Delta p/p  < $5 \% \\
  & (limited by the aperture of the magnets)  \\ \colhline
  Transverse Beam Size   & RMS(x,y) $\approx$ 1 cm  \\
  & (At the entrance face of the LAr cryostat) \\ \colhline
  Beam Divergence & tbd   \\ \colhline
  Beam Entrance Position & Multiple beam windows    \\ \colhline
  Rates & 200 Hz (maximum)    \\ \colhline
\end{cdrtable}





