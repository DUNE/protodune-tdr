%%%%%%%%%%%%%%%%%%%%%%%%%%%%%%%%%%%%%%%%%%%%%%
\section{Run Plan}
\label{sec:runplan}

Preliminary beam simulations show that the hadron rates at 
energies below 1~GeV/c are low. Moreover, low-energy beams are more
subject to degradation by materials in the
beamline.  The optimization of the run plan factors in the beam composition and particle rates of the H4 beamline, and 
also particle interaction topologies in the ProtoDUNE-SP detector. Full FLUKA~%\cite{fluka05,Fluka15}
simulations of particle transport in the ProtoDUNE-SP detector, including the
beam window, have been performed.
% The physics requirement \fixme{that the beam must satisfy overall? Anne} is %the possibility 
% to enable measurement of
% stopping particles and  interactions at both high and low energies.    
%
At a beam momentum of 1~GeV/c, 35\% of protons are stopped before reaching the active TPC region, while the percentage reduces to 0.5\% at 2~GeV/c.  The kinetic energy distributions of protons and pions at the entrance point of the TPC for different beam momenta are shown in Figure~\ref{fig:pandpiint}. 
%The residual energy at the interaction point can be reconstructed by measuring the energy deposited along the proton track.
The fraction of stopping $\pi$'s for one $\pi$
produced at the secondary target is 3\% at $p=0.4$~GeV/c and decreases to 1.3\% at $p=0.7$~GeV/c.
%Having a $\pi$ beam at 0.7~GeV/c still allows 
%measurement of pion interactions down to few tens of MeV with good
%statistics. In a 1-GeV/c beam, the low-energy interactions are %still
%present, albeit at a %with 
%lower rate, as shown in Figure~\ref{fig:pandpiint}.
%
The long distance (37~m) between the secondary target and the front of the LAr cryostat has a significant impact on the pion and kaon rates in the TPC. Given the short pion lifetime, many of the  low-energy pions produced at the secondary target decay in the beam pipe before reaching the cryostat. The situation is even more significant for kaons; most kaons below 2~Gev/c do not make it to the cryostat.
Consequently we will not operate the H4 beamline much below 1~GeV/c in the hadron mode.
For electrons, we would want the beam momentum to go as low as possible to study the topology of very low-energy electron-initiated 
showers.
\begin{cdrfigure}[Energy at interaction]{pandpiint}{Kinetic energy of
    particles at the point of interaction in the ProtoDUNE-SP active
    volume, for different beam momenta. Histograms are normalized to one particle injected in the
    beamline acceptance. FLUKA simulations include the beam window
    materials, beams are considered as monochromatic and
    parallel. Left: protons, Right: pions.}
  \includegraphics[width=0.49\textwidth]{pvarie_intene.pdf}
  \includegraphics[width=0.49\textwidth]{pivarie_intene.pdf}
\end{cdrfigure}
%% end of   part that can go either here or in the run plan 

To formulate a preliminary run plan, we assume the hadron beam spectrum and rates are as given in Tables~\ref{tab:beampartcomp} and~\ref{tab:beampartrates}.   For the purpose of estimating the sample composition and beam time request, the following assumptions are used:
\begin{itemize}
\item { Trigger/date rate = 25~Hz}
\item { Two 4.8 sec spills per SPS Super Cycle }
\item { SPS Super Cycle = 48 sec}
\item { $10^6$ secondary particles on target per spill (hadron beam)}
\item { Particle ID trigger for electrons from 0.5 to 7 GeV/c}
\item { Trigger rate for electron in hadron beam is prescaled to 0.5~Hz}
\end{itemize}
We plan to run the H4 beamline in two modes: the first configuration is optimized for the production of hadrons and the second configuration is optimized for the production of high purity electrons. Even in the hadron mode, the beam is still dominated by electrons, especially for low beam momenta. However, the electrons in the hadron beam are not particularly ``clean'' due to the amount of materials in the beamline from the particle identification (PID) instrumentations .  The proposal is to heavily prescale the electron events using PID (e.g. Threshold Cherenkov counters) trigger while running in hadron mode. The PID systems that contribute significantly to the material budget will be removed when we reconfigure the beamline for electron beam.  We are exploring various run plan scenarios. One of the preferred scenarios is shown in Tables~\ref{tab:HadRunPlan} and ~\ref{tab:ElecRunPlan}. Tables with similar values are expected for the negative beam sample. 
\begin{cdrtable}[Run Plan]{ccccccccc}{HadRunPlan}{A preliminary run plan for ProtoDUNE-SP hadron beam. The expected sample (positive beam) as a function of momentum is shown. }
P & \# of  &\# of $e^+$ & \# of $K^+$ & \# of $\mu^+$ & \# of $p$ & \# of $\pi^+$ & Total \# & Beam Time \\ 
(GeV/c) & Spills  & &  &  &  &  & of Events & (days) \\ \toprowrule
1 & 70K & 84K & $\approx$ 0 & 13K  & 672K & 504K & 1.3M & 19 days\\ \colhline
2 & 20K & 24K & 8K & 21K     & 336K & 480K & 0.9M &5.6 days\\ \colhline
3 & 12K & 14K & 14K  & 14K   & 163K & 516K  & 720K & 3.3 days\\ \colhline
4 & 10K & 12K & 23K & 15K    & 90K  & 460K & 600K & 2.8 days\\ \colhline
5 & 10K & 12K  & 25K  & 6K   & 81K  & 475K & 600K & 2.8 days\\ \colhline
6 & 10K & 12K & 34K  & 5K    & 82K  & 468K & 600K & 2.8 days\\ \colhline
7 & 10K & 12K & 34K & 7K     & 80K  & 467K & 600K & 2.8 days\\ \toprowrule
Total & 142K & 170K & 132K & 81K & 1.5M & 3.4M & 5.3M & 39 days\\
\end{cdrtable}
\begin{cdrtable}[Run Plan]{cccc}{ElecRunPlan}{A preliminary run plan for ProtoDUNE-SP electron beam. The expected sample for positive beam configuration is shown. }
Momentum Bins & \# of Spills per Bin & \# $e^+$ per Bin & Beam Time per Bin \\ 
(GeV/c) & & & (days) \\ \toprowrule
0.5, 06, 0.7, 0.8, 0.9, 1, 2, 3, 4, 5, 6, 7 & 5000 & 300K & 1.4 \\
\end{cdrtable}

Based on the current information available, the total estimated beam time needed to carry out the physics program in this proposal with the assumptions stated earlier is on the order of 16 weeks.
 
 



