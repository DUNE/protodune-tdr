\section{Event simulation}
\label{sec:larsoftsim}

Two kinds of events must be simulated in the ProtoDUNE-SP detector
geometry: beam events and cosmic-ray events.  Beam events are
generated using a dedicated particle gun generator that has as input
parameterizations of the flux and the beam profile parameters.
The upstream beam instrumentation devices -- wire chambers, Cherenkov
counters, and time-of-flight-counters -- are fully simulated as described in
Section~\ref{sec:beaminstruments}. Cosmic-ray
events are simulated either with the CRY~\cite{cry, cry2} event generator or
CORSIKA~\cite{Heck:1998vt}.  Neutrino scattering events are simulated
using GENIE~\cite{Andreopoulos:2009rq}.  While neutrino scattering events will be very
rare in the detector, the extrapolation of the performance of
the detector to the FD will require simulating neutrino scattering
events.  A dedicated generator in LArSoft simulates
radionuclide decay products which can be overlaid on other events.

The detector geometry is coded in GDML files~\cite{Agostinelli:2002hh} that are
generated by the GeGeDe~\cite{gegede} 
geometry system.  These files
contain the locations, sizes, shapes, and material content of the
detector components, the active liquid argon volume, and the
surrounding materials, such as the field cage, the beam windows, the
cryostat the supporting structure, and the experimental hall.  These
external features will impact the distributions of cosmic-ray
particles impinging on the active detector.  The channels and volumes
are numbered and named in the GDML files, with conventions followed by
the LArSoft simulation code.

The active volume of the detector is divided into cubes 300~$\mu$m on
a side, called \textit{voxels}.  GEANT4 tracks particles through the argon.
Each step ends on a voxel boundary, allowing the simulation of small-scale
physics processes such as delta-ray emission and showering at a level
of detail smaller than the intrinsic resolution of the detector.   While GEANT4
calculates the energy deposited by each particle for each step, the
simulation of ionization and scintillation photon emission 
is performed using one of two algorithms in LArSoft:  a dedicated parameterization that
depends on the electric field in the liquid argon and the ionization
density~\cite{Birks:1964zz}, or  NEST~\cite{Szydagis:2011tk}, which is tuned to
previous noble-liquid experimental results and introduces an
anti-correlation between the photon yield and the ionization electron
yield for each step.  

An alternate simulation based on FLUKA~\cite{Fluka15, Ferrari:2005zk, Battistoni:2009zzb} 
is being interfaced into LArSoft.
FLUKA provides a MC simulation of neutrino-nucleus interactions as well as
detailed modeling of particles traversing the detector, configurably replacing the
functionality of both GENIE and GEANT in the simulation.  The availability of alternatives
allows for better flexibility in tuning the models to the data, as well as a basis of
estimating systematic uncertainty.

The average specific energy loss for a minimum-ionizing particle (MIP)
is approximately 2.12\,MeV/cm.  The $W$-value for ionization is 23.6~eV
per electron-ion pair, and the $W$-value for scintillation is 19.5\,eV
per photon, resulting in tens of thousands of drifting electrons and
photons per cm of charged-particle track in the detector.  It is
impractical to simulate the paths of each of these electrons and 
photons using GEANT4, and \fixme{therefore?} computational techniques are incorporated
into LArSoft to achieve a high simulation speed while preserving
accuracy.  The electrons are propagated by LArSoft-specific tools,
including a tool that integrates over the distributions created by longitudinal
and transverse diffusion, 
and a geometry tool that indicates the wire locations that record charge, assuming uniform wire spacing.
\fixme{The prev sentence replaces this:
The electrons are propagated by LArSoft-specific tools
including an integral over the distributions created by longitudinal
and transverse diffusion, and the wire locations on which to record
the charge passing or collecting are looked up from the geometry
assuming uniform spacing. 
}
The effect of charge loss due to attachment
of electrons to impurities (the effect of the electron lifetime), \fixme{that affects the lifetime?}  is
implemented in this step.  
Effects due to space charge are simulated using a smoothly parameterized
map of
distortions in the apparent position components $(x,y,z)$ of the charge
deposits as
functions of the true position.
This map can be made using SPaCE~\cite{Mooney:2015kke}, a
program that traces particle trajectories in LAr based on the
electric field calculated using Poisson's equation, a given
space-charge density map, and the boundary conditions provided by the
cathodes, anodes, and field cages.  It is anticipated that the
space-charge distortions in ProtoDUNE-SP may be as large as
20\,cm~\cite{Mooney:2015kke}. \fixme{I took this out because I don't think it's discussed in this document. Anne``Space charge effects and resulting field distortions for protoDUNE are discussed in Section~\ref{sec:SCEintro}.''}


Photon propagation is simulated using a library that contains the
probabilities of observing a photon emitted at a particular point in
space by a particular photon detector.  Here, the space is divided
into cubical voxels 6\,cm on a side, and the library is indexed by photon
detector element. 
% For the far detector, the lookup library model may
%not scale appropriately, taking too much memory to store
%smoothly-varying photon visibility functions, and parameterizations
%may be used for light propagating from events far from the photon
%detectors, and looked up in the library for light originating from
%points closer to the photon detectors.

The simulated arrival times and charge amounts on each wire, and of
each set of photons arriving at each photon detector, along with the
identity of the particles generating them, are stored in the
simulation output file for use in determining the performance of the
downstream reconstruction algorithms.  These charge depositions and
photons are inputs to the detector response functions -- the field
response and the electronics response are convoluted with the true
arrival times to make simulated waveforms.  The detector field
response functions are simulated using GARFIELD~\cite{garfield}, but
they will be validated with real data, as the simulation contains
oversimplifications, such as inadequate modeling of induction signals.  

The electronics gain is applied so that the
simulated signals match the expected responses.  Simulated noise is
then added, and the result is quantized to reproduce the behavior of a
12-bit ADC, including realistic pedestals and saturation.  A similar
process is followed to simulate the response of the photon detectors,
given the arrival times of the photons.  Functionality exists within
LArSoft to overlay MC-simulated particles with raw digits in
the data in order to simulate pileup of cosmics and other beam
interaction particles. The simulated raw digits are then written to
compressed ROOT files for further analysis.

