\section{Overview}

This chapter of the Technical Design Report is based on the following two documents:
\begin{itemize}
\item DocDB 914 --  \textit{Computing Model for the DUNE Experiment}  which contains the
description of the \pd Computing Model as an Appendix section

\item DocDB 1212 -- \textit{Design of the Data Management System for the
protoDUNE Experiment}

\end{itemize}

\noindent This chapter defines the technical design of the online and offline systems.
The data rate and the total data volume will be the defining factors influencing the design choices and scale of \pd
online and offline computing systems.   At the time of writing, the measurement program, the beam characteristics, the beamline
instrumentation, and many other parameters are being better defined, and thus precise estimates of the data rate and the total
data volume are not yet available.  In the interim, ranges of estimates are given and this chapter presents a small number of
realistic scenarios for the system configuration, scale and performance.

The following components comprise the computing system:
\begin{itemize}

\item The online data storage and management system which interfaces the DAQ and which is tasked with buffering
the data and transporing it to mass storage at CERN and beyond, as well basic monitoring of the data as it is collected.

\item The offline system for data distribution, processing, and analysis on the Grid.  The steps envisaged for data processing are: 
calibration, reconstruction, ntuplizing and user analysis.  Multiple passes through this chain will be required for final results as
calibrations, algorithms, and ntuple definitions are expected to evolve.
A fraction of the data will be subject to \textit{prompt processing}, which performs partial reconstruction of the data for QA/QC purposes
with a short turnaround time (see\,\ref{sec:prompt_processing}).

\item  A Metadata and file catalog system.
\end{itemize}


% Section\,\ref{sec:tpc-signal-calibration} describes the current approach to noise filtering and signal calibration that is being
% developed for DUNE and which will find its application in \pd.


