\section{Overview}



%\noindent 
This chapter outlines  the technical design of the online and offline computing systems.
The data rate and the total data volume will be the main factors influencing the design choices and scale of the \pd
online and offline computing systems.   

%At the time of writing, a few estimates are available based on realistic data-taking scenarios.
The \pd computing comprises the following components: 
\begin{itemize}

\item The online data storage and management system which interfaces to the DAQ. It is responsible for buffering
and transporting the data to mass storage at CERN and beyond \cite{comp_model}.

\item The offline system for data distribution, processing, and analysis on the Grid \cite{data_managm_sys}.  
The steps envisaged for data processing are: 
calibration, reconstruction, ntuplizing and user analysis.  Multiple passes through this chain will be required for final results as
calibrations, algorithms, and ntuple definitions are expected to evolve.
A fraction of the data will be subject to \textit{prompt processing}, which performs partial reconstruction of the data for QA/QC purposes
with a short turnaround time (see\,\ref{sec:prompt_processing}).

\item  A Metadata and file catalog system.
\end{itemize}




\fixme{Thomas: check overlap with DAQ section.}
- 
\fixme{Maxim: just did, I propose that Karol cleans up the DAQ section and refers to this material instead.}