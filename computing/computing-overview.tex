\section{Overview}

This chapter of the Technical Design Report is based on documents and plans developed earlier, in particular
on the \textit{Computing Model for the DUNE Experiment} (DocDB 914) -- which contains the description
of the \pd Computing Model as an Appendix section, and \textit{Design of the Data Management System for the protoDUNE Experiment}
(DocDB 1212). The aim was to capture and present the most important characteristics of the online and offline systems to be built, in a concise manner,
and to also present information developed more recently.

Data rate and volume will be the defining factor for the design choice and scale of \pd online and other computing systems.
Depending on the plan of measurements for \pd
(in development at the time of writing), beam characteristics that will be realized
and many other parameters precise estimate of data rates and volume aren't available just yet. However, it is possible to create ranges
of estimates based on a few realistic scenarios, and translate this into requirements for the system configuration, scale and
performance.

The overall design for the protoDUNE computing can be described as consisting of two major components, which are related:
\begin{itemize}
\item The online data storage and management system which interfaces the DAQ and which is tasked with buffering
the data and transporing it to mass storage at CERN and beyond.
\item The offline system for data distribution and processing on the Grid (e.g. calibration, production and reconstruction). A special case
of such processing is the so-called \textit{prompt processing}, which aims to perform partial reconstruction of the data for QA/QC purposes
with a short turnaround time and can be considered a more advanced type of monitoring.
\end{itemize}

\noindent Crucial for operation of both components is a Metadata and file catalog system.

Section\,\ref{sec:tpc-signal-calibration} describes the current approach to noise filtering and signal calibration that is being
developed for DUNE and which will find its application in \pd.


