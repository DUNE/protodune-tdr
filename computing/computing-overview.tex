\section{Overview}

This section outlines the technical design of the offline computing
system and introduces the software that will be used to simulate and reconstruct ProtoDUNE-SP data.


The data rate and volume will be substantial  over the relatively short
run and this drives many design choices for the offline computing
system.  The system provides resources necessary for data
distribution, processing, and analysis on the
grid~\cite{data_managm_sys}.

All raw data will be saved to tape after being transferred to central
CERN and FNAL computing facilities (Section~\ref{sec:raw_concept}).
During this stage, the metadata and storage locations for all raw data files
will be captured in a file catalog system.

A small portion of the data will be immediately processed for  
data quality monitoring (DQM) purposes (see Section~\ref{sec:prompt_processing}).
%
The processing steps on the full data sample (see
Section~\ref{sec:prod-process}) include ADC-level corrections,
potential excess-noise filtering, signal processing, reduction,
calibration, reconstruction, summarizing, and final user analysis.
Multiple passes through this chain will be required for final results,
as calibrations, algorithms, and summary definitions are expected to
evolve.  

Details on the software framework, event simulation and
reconstruction are covered toward the end of this section.

