\section{Prompt Processing}
In the present context, \textit{Prompt Processing} means a number of fast reconstruction processes (also called
``express stream'' sometimes) which process a fraction of raw data. Its main purpose is to aid in QA of the data
and produce quick calibrations which may be necessary for high quality monitoring of the detector. It is understood
that a limited number of metrics will be calculated to make the proccess as quick as possible in order to enable
researchers to take action should the QA process indicate a potential problem in the detector or the data.

The following parameters of prompt processing need to considered:
\begin{itemize}
\item Desired turnaround time.
\item Location of the computing resources utilized.
\item Fraction and type of data to be processed in this manner.
\end{itemize}

