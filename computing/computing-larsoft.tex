\section{The LArSoft framework for simulation and reconstruction} 
\label{sec:comp:larsoft}

%\fixme{cross check with description in DAQ section;  there is some tolerable (?) overlap. (looks ok Anne}
LArSoft~\cite{larsoft-web} is a suite of tools for simulating and
reconstructing data collected from LArTPC detectors.  It is
built on the \textit{art}~\cite{art-web} event-processing framework.  The
main features of the \textit{art} framework are 
its configurability by
human-readable and editable control files (that use the Fermilab
Hierarchical Control Language (\textit{FHiCL})), and
the scheduling of 
program module execution. The modules are of five types: event sources, filters,
data-product producers, analyzers, and output.  
Common utilities that
can be accessed by any program module at any time are called \textit{services}.

The \textit{art} framework defines the input/output structure of
ROOT-formatted files using TTrees to store the data, metadata, and
provenance information.  The provenance information consists of the
contents of the FHiCL documents used to steer the processing of the
job that created the output file, and those of input files and parents.  
%\fixme{give some adjective before `file' e.g., `output'?}

The \textit{ art} framework's division of the simulation and reconstruction
jobs into modular pieces allows multiple developers to contribute to an
effort, and to test their ideas in isolation before integrating them
into a larger system.  Because the data read in from an event is
placed in read-only memory, analyzers can program with confidence that
upstream algorithms cannot alter the data, but must produce additional
data products which can later be processed or written out.

The LArSoft suite provides the interface to the event generators and
Geant4~\cite{Agostinelli:2002hh} 
for simulation of the passage of particles
through the detector, the details of which are described in
Section~\ref{sec:larsoftsim}, and event reconstruction, the details of
which are presented in Section~\ref{sec:larsoftreco}.  

The \textit{art}
framework and LArSoft source code are publicly available, and pre-built
versions are provided~\cite{scisoft-web} for supported versions of Linux and Mac
OS~X.  Tools for compiling the framework and
applications are also provided, along with all of the required
dependencies, including the gnu C++ compiler.  The versions of the
software and its dependencies are managed by the UNIX Product Support (UPS) system,
which allows easy version selection, setup and configuration of the LArSoft
environment on computers with previous versions already installed, such as
those at Fermilab.  

LArSoft is under rapid development by both the
core LArSoft team and by contributors from participating experiments:
ArgoNeuT, LArIAT, MicroBooNE, SBND, and DUNE.  Within DUNE, the LArTPC
near-detector option, both ProtoDUNE detectors, and the Far Detector
are clients and contributors of LArSoft.