\section{Prompt Processing for Data Quality Monitoring}
\label{sec:prompt_processing}

As described in Section~\ref{sec:daq_online_monitoring}, the first
point at which data quality monitoring occurs is directly inside the
DAQ Online Monitoring (OM).  The DAQ computing cluster hardware is
relatively high performance and has access to the full, high-rate data
stream.  As such it is ideal for monitoring algorithms which require
small amounts of CPU and a large fraction of the data.

On the other end of this spectrum, some monitoring algorithms have
large CPU requirements but produce meaningful feedback on relatively
little data.  Running these algorithms on commodity cluster hardware
is more cost effective.  To manage these jobs, a special purpose system
called the ``protoDUNE prompt processing system'' (p3s) is developed.
Unlike traditional batch systems it limits maximum latency to provide
results at the cost of 100\% data throughput.  The p3s is portable to many 
native batch systems and is fully
  user-configurable.  It can run multiple independent sets of multiple
  interdependent jobs, and run them in parallel -- to the
  extent allowed by available CPU resources and while satisfying dependency information.

The prompt processing is expected to sample about 1\% of the most
immediate data just after it is saved by the DAQ and is available to the
hardware on which it runs.  An initial estimate finds that at least
300 dedicated cores will be required to achieve this.  This estimate
must be refined as a comprehensive list of monitoring algorithms is
developed.

The prompt jobs are developed in the familiar form of offline software
modules to LArSoft (Section~\ref{sec:comp:larsoft}).  The processing is expected to include algorithms
from signal processing through to full reconstruction.  Due to the
sharing of the underlying art framework in both prompt processing jobs and
DAQ OM modules, it will be easy to migrate algorithms between the two
contexts in the case of computer-hardware resource constraints.

%%%%%%%%%%%%%%%%%%%%%%%%%%%%%%%%%%%%%%%%%%%%%%%%%%%%%%%%%%%%
\section{Production processing}
\label{sec:protodune-offline}

The second major user of the raw data is the production processing.
It will make several passes of 100\% of the raw data over time as
algorithms improve.  It has two major stages, data reduction and 
event reconstruction, which finally feeds into user analysis.

%%%%%%%%%%%%%%%%%%%%%%%%%%%%%%
\subsection{Data reduction}
\label{sec:datareduc}

The initial steps of production processing can be factored out into a
specific stage in order to allow a vastly reduced data set to be
produced which retains all signal information.  This data reduction
stage consists of about 20\% of the total production processing CPU
time.

It consists of four major steps.  First, it applies any ADC-level
corrections followed by a software noise filter if excess noise is
encountered.  These first two steps alone will allow a
$6\times - 8\times$ compression factor due to reduction of information
entropy that results from removal of excess noise.  The third step applies signal
processing.  This crucially deconvolves the detector field response
functions from the waveforms and identifies the signal regions of
interest (signal-ROI).  In particular, this allows the otherwise small
induction-plane signals to become unipolar and rise above the inherent
noise floor.  By definition, the signal-ROI selection rejects portions
of waveforms that are indistinguishable from noise and thus carry no
signal information.  It is this selection that provides the bulk of
the reduction.  Finally, the oversampling of the data is no longer
required and is removed while keeping safely below the Nyquist limit
and the data is packed, compressed and saved to file.  This reduction
targets only the TPC data; all other raw fragments are simply
copied to the output file.

The reduced TPC data is more than 400 times smaller than the raw
input. This translates to reducing the 3\,PB of data described in
table~\ref{tab:goldi} to only 6.5\,TB.  This relatively small data set
can then be more easily transferred to the remaining, high-CPU part of
the production processing chain and it facilitates sharing of the data
in general.  Independent of the output of the data reduction, the full
raw data will be still archived to tap at CERN and FNAL as described
in Section~\ref{sec:raw_concept} so that as data reduction techniques
improve they can be rerun, and if any special studies require it, they
may have access to the full raw waveforms.



%%%%%%%%%%%%%%%%%%%%%%%%%%%%%%
\subsection{Reconstruction processing}
\label{sec:prod-process}

Starting with the signal-ROI there are two basic approaches to
reconstruction, which are described in detail in the following
sections.  The first starts with fitting multiple Gaussian
distributions to the waveforms and the second to retaining their
binned structure.

