%%%  \chapter{introduction}


%%%%%%%%%%%%%%%%%%%%%%%%%%%%%%%%%%%%%%%%%%%%%%%%%%%%

%%%%%%%%%%%%%%%%%%%%%%%%%%%%%%%%%%%%%%%%%%%%%%%%%%%%

\section{ProtoDUNE-SP in the context of DUNE/LBNF}

ProtoDUNE-SP is the single-phase DUNE Far Detector prototype that will be constructed and operated at the CERN Neutrino Platform (NP) starting in 2017. It was proposed to the CERN SPSC in June 2015 (SPSC-P-351), and following positive recommendations by SPSC and the CERN Research Board in December 2015, was approved at CERN as experiment NP-04 (ProtoDUNE). The Fermilab Director and the CERN Director of Research and Scientific Computing signed a Memorandum of Understanding (MoU) for this experiment in December 2015 that is initially valid until December 2022, % in the first instance, 
and may be extended by mutual agreement. 

%ProtoDUNE-SP, a significant experiment in its own right, will have a total liquid argon (LAr) mass of \kt{0.77} and represents the largest monolithic single-phase LArTPC detector to be built to date. %so far. 
%The CERN Neutrino Platform, in an extension to the EHN1 hall in the North Area, will provide a new dedicated charged-particle test beamline. ProtoDUNE SP aims to take its first beam data before the LHC long shutdown (LS2) at the end of 2018.

%ProtoDUNE-SP is a crucial part of the DUNE effort towards the construction of the first DUNE \ktadj{10} far detector module, that will have a total LAr mass of about \kt{17}. It will prototype the designs of most of the DUNE far detector components at a 1:1 scale and with an extrapolation of about 1:20 in total mass size, very similar to the scaling factor adopted in the past by ICARUS from the 10-m$^3$ prototype to the T600 detector (two half-modules of about 375 t total LAr mass each).

ProtoDUNE-SP, a crucial part of the DUNE effort towards the construction of the first DUNE \ktadj{10} fiducial mass far detector module (17\,kt total LAr mass), is a significant experiment in its own right. ProtoDUNE-SP, with a total liquid argon (LAr) mass of 0.77\,kt, represents the largest monolithic single-phase LArTPC detector to be built to date. %so far. 
It will be housed in an extension to the EHN1 hall in the North Area, where the CERN NP will provide a new dedicated charged-particle test beamline. ProtoDUNE SP aims to take its first beam data before the LHC long shutdown (LS2) at the end of 2018.

ProtoDUNE-SP will prototype the designs of most of the single-phase (SP) DUNE far detector components at a 1:1 scale, with an extrapolation of about 1:20 in total mass size,. \fixme{mass and size?} This is very similar to the scaling factor adopted %in the past 
by ICARUS; its T600 detector, split into two half-modules of about 375\,t total LAr mass each, was preceded by a 10-m$^3$ prototype.


The SP detector elements, consisting of the time projection chamber (TPC), the cold electronics (CE), and the Photon Detection System (PDS), are housed in a cryostat that contains the liquid argon target material. The cryostat, a free-standing steel-framed vessel with an insulated double-membrane system, is based on the technology used for liquefied natural gas (LNG) storage and transport. % ships. 
%A cryogenics system keeps the LAr at a stable temperature of about 89\,K through a process of recovering evaporated argon, recondensing it and returning it to the cryostat via a closed loop for forced recirculation of the liquid through the O$_2$ and H$_2$O filtration system that ensures the required LAr purity. 
A cryogenics system maintains the LAr at a stable temperature of about 89\,K and at the required purity level through a closed-loop process that recovers the evaporated argon, recondenses and filters it, and returns it to the cryostat. 


The TPC includes the Anode Plane Assemblies (APA), the Cathode Plane Assemblies (CPA), and the field cage (FC), all components identical in design to those for the Far Detector. It consists of two arrays of three 6-m$\times$2.3-m wire-wrapped APAs at the opposite sides of the central CPA. Each APA is made of three parallel planes of wires (4.5 mm pitch, 2,560 wires) 
\fixme{the details in parentheses seem unnecessary (and distracting) here.}
oriented at different angles with respect to each other. The CPA is held at $-$180\,kV providing the 500-V/cm drift field in the 3.6-m drift regions between the CPA and APAs. % (also identical to the Far Detector configuration). 
Uniformity of the electric field is guaranteed by the FC that delimits the volumes between CPA and APA planes.

The CE, mounted onto the APA frame and immersed in LAr, amplifies and continuously digitizes the induced waveforms on the sense wires at several MHz, and transmits these data to the Data Acquisition system (DAQ). The data are then transmitted through the buffer to disk, then to the central CERN Tier-0 Computing Center and finally to other partner sites for processing and analysis.  

The current PDS reference design uses a thin radiator coated with a wavelength-shifting layer of tetraphenyl-butadiene (TPB) that converts incoming VUV (128 nm) %Ar 
scintillation photons to longer wavelength photons, in the visible blue range. 
\fixme{previous sentence would read better with `light' rather than `photons'} The radiator is placed in front of a doped acrylic bar 200 cm long and 7.6 cm wide. Half of the converted photons will be emitted into the bar. A fraction of the wavelength-shifted optical photons are then internally reflected to the bar's end where they are detected by silicon photomultipliers (SiPMs).
The APA frame is designed with ten bays into which PDS modules (coated bars) can be inserted. The bars are inserted into the frames after the TPC wires have been strung, allowing final assembly at the integration area at the CERN NP prior to installation inside the cryostat. 

%After completion of the detector mounting, LAr filling and commissioning of the prototype detector, 
Once the prototype detector is commissioned, the charged-particle beam test will provide critical calibration measurements necessary for precise calorimetry as well as invaluable data sets for optimizing the event reconstruction algorithms -- i.e., for finding interaction vertices and for particle identification -- and ultimately for quantifying and reducing systematic uncertainties. These measurements are expected to significantly improve the physics reach of the DUNE experiment.

Other important goals for ProtoDUNE-SP include validation of the membrane cryostat technology and associated cryogenics, and of the networking and computing infrastructure that will handle the data and simulated data sets.
%ProtoDUNE-SP is thus an indispensable step to validate and benchmark the DUNE far detector engineering, to provide a well characterized charged-particle beam data set and to validate and test associated infrastructure requirements.  Given its technical challenges, its importance to the DUNE experiment and the timeframe in which it must operate, ProtoDUNE-SP requires a strong organizational structure and a collaborative effort from most of the DUNE Collaboration including US National Laboratories and University groups, CERN and international partners in the EU and Latin America. 
The construction and operation of ProtoDUNE-SP are thus indispensable steps toward the first DUNE Far Detector module. ProtoDUNE-SP will allow the collaboration to validate and benchmark the far detector engineering, to validate and test associated infrastructure requirements, and to extend the physics reach of the experiment. 

Given its technical challenges, its importance to the DUNE experiment and the timeframe in which it must operate, ProtoDUNE-SP requires a strong organizational structure and extensive participation from the DUNE collaboration.
\fixme{Is the following needed? Seems like it potentially excludes some regions. `', including U.S. National Laboratories and University groups, CERN and international partners in the EU and Latin America.'' }

%In parallel, another ProtoDUNE (ProtoDUNE-DP) of identical size based on the dual-phase LArTPC technology will be built at CERN, as complementary part of the DUNE strategy for reaching the design goal of a 40 kton fiducial detector mass. ProtoDUNE SP and ProtoDUNE DP share the CERN Neutrino Platform: this offers great opportunities for expanding common ``infrastructures" (e.g. cryo-system and LAr purification, cryo-instrumentation, on-line computing farm, off-line computing resources, ...) and for an optimal use and sharing of resources.
In parallel, a second prototype of identical size but based on the dual-phase (DP) LArTPC technology will be built at CERN since one or more of the subsequent Far Detector modules may use the DP design. Called ProtoDUNE-DP, it complements ProtoDUNE-SP in the DUNE strategy for reaching the design goal of \ktadj{40} fiducial mass. ProtoDUNE-SP and ProtoDUNE-DP share the CERN NP; this offers great opportunities for expanding common infrastructures (e.g., systems and instrumentation for cryogenics and LAr purification, an online computing farm, offline computing resources, and so on) and for an optimal use and sharing of resources.

The ProtoDUNE-SP detector is similar in design to the Short Baseline Near Detector (SBND), the near detector of the SBN experiment at Fermilab, and the timeline of the two experiments is comparable. The development of effective synergies, the exploitation of common detector solutions and common test tooling, and the optimal use of resources (human and financial) are all goals of the on-going DUNE and SBN management efforts. 

%%%%%%%%%%%%%%%%%%%%%%%%%%%%%%%%%%%%%%%%%%%%%%%%%%%%
\section{ProtoDUNE-SP Management and Organization}

%In the DUNE organizational chart, protoDUNE-SP is placed at the same high level as the Far Detector and the protoDUNE-DP organizations. This fosters a strong coupling with the Far Detector working groups and their activities as well as an efficient  overlap between the two protoDUNEs. It should be noted, however, that the protoDUNEs (SP and DP) are identified as distinct teams that take ownership of and responsibility for their separate experiments.
 
Within DUNE, the ProtoDUNE-SP and DP organizational structures are placed at the same high level as that for the Far Detector. This fosters a strong coupling with the Far Detector working groups and their activities as well as an efficient 
 overlap between the two ProtoDUNEs. It should be noted, however, that the two ProtoDUNEs are identified as distinct teams that take ownership of and responsibility for their separate experiments.
 
CERN and the NP organization are %fundamental partners for the 
partnering with DUNE in the extension of the H4 (and H2) beamlines to ProtoDUNE SP (and ProtoDUNE DP) and for the delivery of the cryostat and the cryogenics infrastructure.

The ProtoDUNE-SP leadership is the primary contact to the DUNE management as well as the interface with the CERN teams responsible for the NP, and between CERN and the DUNE collaboration in all matters relating to ProtoDUNE-SP.

ProtoDUNE-SP is organized into working groups dedicated to specific tasks. The responsibilities for construction of the TPC,  the Cold Electronics read-out boards and the Photon Detector System fall under %responsibility of 
the respective Far Detector Working Groups. Specifically appointed ProtoDUNE-SP \textbf{TPC}, \textbf{PDS} and \textbf{Electronics} teams take on the responsibility for the timely and technical success of component assembly, testing and commissioning at CERN.

A large international team within the ProtoDUNE-SP organization is dedicated to the \textbf{DAQ system}. 
The other %relevant 
detector components, i.e., the \textbf{HV System},  \textbf{Cryogenic Instrumentation} and \textbf{Slow Control System}, are developed jointly by ProtoDUNE-SP and ProtoDUNE-DP teams, and all include significant contributions from the CERN NP.

Teams dedicated to an external \textbf{Cosmic Muon Tracker} (CRT) and to \textbf{Beam Instrumentation} are responsible for design, construction, commissioning and operation of corresponding hardware devices and for the DAQ interface.

A ProtoDUNE 
\fixme{SP or for both SP and DP?} team at CERN is dedicated to \textbf{Software Development} in connection with the LArSoft collaboration at Fermilab. This team supports the offline analysis activity of the \textbf{Measurement} WG in the area of fast delivery of detector performance characterization and physics results based on beam and cosmics data.

\textbf{Installation} and \textbf{Engineering Integration} teams will provide the interface to CERN in all matters related to the infrastructures in the assembly hall and in the experimental hall, including the cryostat and cryogenics \& LAr purification systems, which are the responsibility of
the CERN NP.

%%%%%%%%%%%%%%%%%%%%%%%%%%%%%%%%%%%%%%%%%%%%%%%%%%%%
\section{Goals of ProtoDUNE-SP}

%The primary goal of the ProtoDUNE-SP test program at CERN is very clearly a twofold one: First,  to benchmark and, if found fully adequate, to endorse the main technical solutions for the DUNE far detector components, and secondly, to perform the measurements needed to understand and possibly quantify the systematic uncertainties that will affect the DUNE oscillation measurements. The latter goal anticipates physics outcomes relevant on their own.
The ProtoDUNE-SP test program at CERN has two very clear primary goals. First, it will benchmark the principal technical solutions for the DUNE SP far detector components and, if they are found to be fully adequate, to endorse them. Secondly, it will perform the measurements needed to understand, and quantify to the extent possible, the systematic uncertainties that will affect the DUNE oscillation measurements. This second goal anticipates physics outcomes that will be relevant independent of the Far Detector.

%The detector operation in real experimental conditions and for an extended run period will allow for a full characterization of all the components, from the membrane cryostat and the cooling and purification circuit, to the APA design and its read-out cold electronics layout and HV system, to the photon detection system and its read-out warm electronics.
Operating the detector in real experimental conditions and for an extended period will allow for a full characterization of the components, including the membrane cryostat and the cooling and purification circuit, the APA design and the layout of its cold read-out electronics, the HV system, and the PDS and its warm read-out electronics.

The use of a well defined test beam of charged particles of known type and incident  energy will significantly enhance the understanding of the ultimate performance of the LArTPC technology and boost the optimization of event reconstruction, particle identification (PID) algorithms and calorimetric energy measurements.  The beam measurements will serve both as a calibration data set to tune the Monte Carlo simulations and as a reference data set for the DUNE experiment. 

%Pion and proton beams from around one to a few GeV will be used primarily to study hadronic interaction mechanisms, secondary particle production and, at higher energies, shower reconstruction and energy calibration. Electrons will be used to benchmark and tune electron/photon separation algorithms, to study electromagnetic cascade processes and to calibrate electromagnetic showers at higher energies. Charged kaons produced in the tertiary beam line are rare but are copiously produced by the pion beam interactions inside the detector. These will be extremely useful to characterize kaon identification efficiency for proton decay sensitivity studies.  Samples of stopping muons with Michel electrons from muon decay (or without, in case of negative muon capture) will be used for energy calibrations in the low energy range of the SN neutrino events and for the development of charge-sign determination methods. 
Pion and proton beams in an energy range from about one to a few GeV will be used primarily to study hadronic interaction mechanisms and secondary particle production.  At higher energies, these beams will be used to study shower reconstruction and energy calibration. Electrons will be used to benchmark and tune electron/photon separation algorithms, to study electromagnetic cascade processes and to calibrate electromagnetic showers at higher energies. Charged kaons produced in the tertiary beamline are rare but are copiously produced by the pion beam interactions inside the detector. These will be extremely useful for characterizing kaon identification efficiency for proton decay sensitivity studies.  Samples of stopping muons with Michel electrons from muon decay (or without them, in the case of negative muon capture) will be used for energy calibrations in the low-energy range of the SN neutrino events and for the development of charge-sign determination methods. 

A cumulative ProtoDUNE-SP test-beam run period of eight weeks is assumed, but it depends on the extent of beamline sharing with other users at EHN1. The run will take place prior to the long shutdown of the LHC in late 2018 (LS2). 

%During no-beam time cosmic data will be acquired. A dedicated external trigger system consisting of arrays of scintillator paddles, suitably positioned and arranged in coincidence trigger logic, will select specific classes of cosmic muon events. Dedicated runs, e.g. long muon tracks crossing the entire detector at large zenith angles allow for an overall test of the detector performance and the DAQ. Muons stopping inside the LAr volume and the accumulation of accurate Michel electron spectra may serve for energy calibration purposes in the low-energy range.
ProtoDUNE-SP will acquire cosmic data during periods with no beam. A dedicated external trigger system consisting of arrays of scintillator paddles, suitably positioned and arranged in \textit{coincidence} trigger logic, 
will select specific classes of cosmic muon events. Dedicated runs, e.g., runs looking for long muon tracks crossing the entire detector at large zenith angles, allow for an overall test of the detector performance and the DAQ. Runs looking for muons stopping inside the LAr volume and the accumulation of accurate Michel electron spectra may be useful for energy calibration purposes in the low-energy range.

%It is important to note that, besides calibration purposes and detector performance characterization, the unprecedented event reconstruction capability of the LArTPC technology combined with the large active volume of the ProtoDUNE-SP detector exposed to the CERN charged-particle beams open the way to a truly rich program of new physics investigations on particle interaction processes. The LArTPC features simultaneously precise tracking (3D imaging detector) and accurate measurement of energy deposited (homogeneous calorimeter). The large size of the active volume in ProtoDUNE allows for good containment of the hadronic and electromagnetic interaction products in the few GeV range. No other detector ever had all these features combined in one. 

It is important to note that ProtoDUNE-SP offers much beyond calibration  and detector performance characterization.  The LArTPC simultaneously features precise 3D tracking and accurate measurement of energy deposited. Its large active volume allows for good containment of the hadronic and electromagnetic interaction products in the few GeV range. These capabilities have never before been combined in one detector.  The unprecedented event reconstruction capability combined with the exposure of the detector's large active volume to the CERN charged-particle beams open the way to a truly rich program of new physics investigations into particle interaction processes. 


% Hadroninc ($\pi$, K and $p$) interactions on an Ar target around 1 GeV produce low multiplicity final states rather than "hadron showers",  and 1 GeV-electrons  (critical energy $\simeq 30$ MeV in Ar) produce low-populated cascades, with only a few tens of secondary energetic electrons(positrons). "TPC/imaging-aided calorimetric measurements" in this energy range may allow to investigate energy deposition mechanisms and reconstruction methods where the usual hadronic and el.m. shower concepts and features are not well defined or cannot be applied.Calorimetric measurements of the energy deposited can be accomplished,  whenever possible, for each individual secondary particle/track thanks to the imaging capabilities of this type of detector.In particular, the determination of the el.m. content in hadron initiated cascades, $\pi^0$ multiplicity and the energy fraction carried as a function of primary hadron incident energy will be of interest.


 Hadroninc ($\pi$, K and $p$) interactions on an Ar target around one GeV produce low-multiplicity final states rather than ``hadron showers,'' 
 and 1-GeV electrons  (with critical energy $\simeq 30$ MeV in Ar) produce low-populated cascades, with only a few tens of secondary energetic electrons (positrons). \fixme{I don't see relationship between this sentence and next. Maybe need a ``therefore'' in next sentence?}
``TPC/imaging-aided calorimetric measurements'' in this energy range may allow investigation of 
energy deposition mechanisms and reconstruction methods where the usual hadronic and electromagnetic shower concepts and features are either not well defined or cannot be applied.
Calorimetric measurements of the energy deposited can be accomplished,  whenever possible, for each individual secondary particle/track thanks to the imaging capabilities of this type of detector.
\fixme{the `wherever possible' makes the sentence meaningless; please clarify}
In particular, the determination of the electromagnetic  content in hadron-initiated cascades, $\pi^0$ multiplicity, and the energy fraction carried as a function of primary hadron incident energy will be of interest.

\fixme{This whole last paragraph is somehow not gripping me; could it be clarified and then maybe put above the previous paragraph (which would make a better conclusion.}
%In conclusion, test beam data will be essential for a rich physics program with ProtoDUNE at the CERN NP.
