The quality control and quality assurance procedures for the individual components of the ProtoDUNE-SP detector are collected in this chapter.

%%%%%%%%%%%%%%%%%%%%%%%%
\section{APA}

Each ProtoDUNE-SP APA will be subject to a testing program during production that will demonstrate their compliance with the design and manufacturing requirements for the experiment.  Given the large size of the APAs, and the highly specialized cryogenic and electronic infrastructure necessary for their designed operation, the testing plan followed during production is necessarily targeted at those features of the APAs which can be addressed prior to final installation in the cryostat.  A separate suite of testing, in a cryogenic environment, is planned to be conducted at CERN, as described in ???? \fixme{is there a section in the TDR for this cold-box test?}.

The tests described in this section will be conducted on APAs during construction, as well as on completed APAs.  Subcomponents ($e.g.$ - wire boards, BeCu wire, etc..) are verified or tested by the appropriate vendor or subsystem manager prior to their use in APA construction.  Each APA will be subject to the testing listed in Table \ref{tab:apatesting}.   Many of the tests (e.g. - tension, continuity, isolation, position survey, bias voltage) are repeated as each successive wire layer of the APA is added, testing the newly installed wires as well as spot-checking wires from previously installed layers. The summary of the tests on each APA, and the associated data collecting during the tests, are recorded in a traveler.

\begin{cdrtable}[APA Testing]{llr}{apatesting}{Tests to be conducted on APAs during production.}
Label & Requirement & Test Method  \\ \toprowrule
Tension      & All wires must be tensioned to 5$\pm$1 N & Laser Survey \\ \colhline
Cryo       &  All APA must operate at 87 K  & Cryotest \\ \colhline
Electrical       & All wires electrically connected at top/bottom of APA  & Continuity \\ \colhline
Isolation & Neighboring APA wires have $\geq$1 M$\Omega$ resistance & Multimeter\\ \colhline
Voltage & Anode planes must hold $\geq$100$\%$ bias & Bias \\ \colhline
Position & All wires must be within 500 $\mu$m of design location & Survey\\ \colhline
Frame & APA frame satisfies all bow/twist/fold criteria & Survey\\ \colhline
\end{cdrtable}

%%%%%%%%%%%%%%%%%%%%%%%%
\section{CPA}

The following activities are planned to assure the CPA meets all design requirements as defined in the fabrication drawings and description in the sections above:
\begin{itemize}
\item Performed 35 ton HV test at FNAL.
\item Fabricate four prototype CPAs to test the design and fabrication and assembly methods.
\item Installation test at Ash River:
\begin{itemize}
\item Test the lifting and handling of the four prototype CPA's.
\item  Load the 4 prototype CPA's with FC modules and test and evaluate their installation.
\end{itemize}
\item Develop a QC plan for inspecting every fabricated part of the CPA frame to make sure they meet the dimensions and tolerances on the fabrication drawings.
\item Develop a QC plan for inspecting and measuring each CPA module and completed CPA plane to ensure they meet the dimensions and tolerances on the drawings.
\item Perform tests of each joint in the CPA frame (see Section 3) to ensure that their design and strength meets the load requirements.
\item Create an integrated model of the entire TPC to evaluate interfaces and installation methods.  
\item Develop a QC for receiving the resistive panels.  Measure the dimensions to confirm they meet the drawings and setup a plan and acceptance criteria to insure that panel resistance is acceptable.
\item Develop a QC  HV test at CERN for evaluating side to side and top to bottom resistance for each completed CPA but after final assembly and after hanging during installation.  
\end{itemize}

%%%%%%%%%%%%%%%%%%%%%%%%%
\section{Field Cage}

The following activities are/will be performed to assure the field cage meets all design criteria as defined in the fabrication drawings and description in the sections above:
\begin{itemize}
\item	Fabricate prototype FCs to test the design and fabrication and assembly methods.
\item	Installation test at Ash River:
\begin{itemize}
\item	Test the lifting and handling of the prototype FCs.
\item	Mount the prototype FCs to the CPA modules and test and evaluate their installation.
\end{itemize}
\item	Develop a QC plan for inspecting every fabricated part of the FC frame to make sure they meet the dimensions and tolerances on the fabrication drawings.
\item	Develop a QC plan for inspecting and measuring each FC module and completed FC plane to insure they meet the dimensions and tolerances on the drawings.
\item	Perform tests of each joint in the FC frame (see Section XX) to insure that their design and strength meets the load requirements.
\item	Create an integrated model of the entire TPC to evaluate interfaces and installation methods.  
\end{itemize}

 
