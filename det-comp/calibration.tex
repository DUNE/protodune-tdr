

\subsection{Cosmic muon tagger}
\label{muontagger}

        While there may be motivations from a reconstruction standpoint for a
cosmic veto, from a calibration standpoint a tracking system to provide known 
cosmic trajectories is of interest. The tracker may also be used for
off-beam triggering on stopped muons, as well as detector triggering during
times when there is no beam.

        Counters for a cosmic tracker are available from spares made for the
Double CHOOZ experiment \cite{mucounters}. Light created in the
counters is transmitted via fibers to a 64-anode PMT and read out via
electronics that use a USB interface.  The counter granularity is roughly 2.5~cm, but
multiple layers of the counters could be used for finer granularity.

\fixme{This is a VERY sparse description of a muon tagger system for a TDR. A picture of a 3D model,
amount of area covered, readout and data stream issues would be good to address at least at some level.
Assume some baseline option and mention that changes/"optimizations" are still being considered.}

        Although there are many possible configurations for deployment of the
counters, a particularly interesting deployment would be fore-and-aft,
therefore triggering on muons that had trajectories similar to beam events. In
this configuration, the counters could also serve as a tagger for beam halo
events.  By tagging such events, only those whose energy and position are
reasonably well constrained will be part of the beam calibration.



%%%%%%%%%%%%%%%%%%%%%%%%%%
%%---\subsection{PDS Calibration} -- ANL -- Zelimir
%%---     Mechanical design -- Zelimir interface with Bob Sands?

%%%%%%%%%%%%%%%%%%%%%%%%%%
%%\subsubsection{PDS Calibration Run Plan}
 %%    Run/Measurement plan  -- Toups ?
  %%   Light yield vs. field
   %%  timing studies,
   %%  beam off trigger

%%%%%%%%%%%%%%%%%%%%%%%%%
\subsection{QC Procedures}

\fixme{Any QC procedures for the cosmic tagger ??}
