
\section{Calibration Systems}
\fixme{from Josh K}
%%%%%%%%%%%%%%%%%%%%%%%%%%
\subsection{Motivation}


	The scientific program of protoDUNE is critical to the ultimate success
of DUNE.  protoDUNE will be the DUNE Collaboration's only opportunity to
measure the response of a DUNE-style LAr-TPC to hadrons (as well as $\gamma$s,
$\mu$s, and electrons), and to compare those measurements to a model of the
detector response. That model can then be used to predict the response for DUNE
itself.  Precision measurements of response in testbeams are a common approach
of long-baseline neutrino experiments. As a recent example, NO{$\nu$}A has
found that some of their measurements are limited by the lack of knowledge of
this response, which a testbeam would have provided.
	
	By itself, however, protoDUNE does {\it not} provide a response
measurement that can simply be mapped onto DUNE.  Rather, protoDUNE provides
measurements of the part of the response that depends on the interactions of
various particle species in LAr.  The remainder of the detector response, which
includes reconstruction resolutions and biases, the effects of noise, space
charge and how it impacts reconstruction, and the conversion from observed
charge to energy, all must be calibrated and ultimately removed from the
physical hadron response.  Misinterpreting, for example, longitudinal diffusion
as a fundamental part of shower development of $\pi^+$s means our understanding
of $\pi$s will be wrong in DUNE---and there will be no test of that response in
the DUNE FD, independent of neutrino events themselves.  By building protoDUNE
in a way that mimics the DUNE far detector, we anticipate that such errors will
`cancel out'; in practice we cannot rely on such a cancellation without knowing
how different the parameters governing the response may be.

	Ultimately, what this means is that we must calibrate a model of the
detector---likely a LArSoft model---and use it to predict the response to
hadrons.  If the detector model is accurate, differences between the predicted
response to various hadrons and measurements allows a correction to the hadron
response that can be used for the DUNE experiment.  

	The existing option for creating a calibrated detector model for
protoDUNE is to presume that calculations of the electric field throughout the
detector, including space charge effects and impurities, are accurate, and that
with this as input the detector response is uniquely calculable throughout the
volume and for all times, up to the details of the electronics transfer
function (and of course the response to various particle species which is what
is being measured by protoDUNE).  

	Figure~\ref{fig:35t} shows the variations in electron lifetime within
the 35 tonne prototype as a function of position, as measured by {\it in-situ}
purity monitors. The clear differences top and bottom indicate that we cannot
assume that measurements in just a few places represent the detector at a
whole. In addition to these variations, changes with the ambient environment
were also seen, indicating that measurements must be made with reasonable
frequency as well.  An idealized view of how an operating detector will behave
would miss such variations.
	 
	The calibration program for DUNE has as its goal both the
measurement of the parameters governing detector response and {\it tests} of
the predicted response.  Table~\ref{tab:params} lists many of the critical

\begin{cdrtable}[Parameters to be calibrated for ProtoDUNE model]{lllll}{params}{Parameters to be calibrated for ProtoDUNE model}
Parameter & Name & Ex-situ Calibration & Calibration & Test \\ \toprowrule
$W$ & Ionization Energy  & Benchtop & None & Cosmics \\  \colhline
$C$ & ADC/Charge Map & Benchtop Pulsers & Front-End Pulsers & Cosmics \\ \colhline
$f$ & Energy Scale & Calculation & Stopped cosmics & Through-going cosmics \\ \colhline
$R$ & Electron recombination & Calculation & Laser & Cosmics \\ \colhline
$\tau$ & Electron lifetime & Purity Monitors & Laser & Cosmics \\ \colhline
$\vec{E}(x,y,z)$ & Electric Field map & Calculation & Laser & Cosmics \\ \colhline
$v_d$ & Drift Speed & Purity+temp+Calc.& Laser & Cosmics \\ \colhline
$d$ & Electron Diffusion & Calculation & Laser & Cosmics \\  \colhline
$\rho_E$ & Field response & Calculation & Photocathode & Cosmics \\
\end{cdrtable}

parameters that need to be measured.  We assume in this note that there will be
several auxiliary systems that already provide measurements or estimates of
some of these parameters:
\begin{itemize}
\item Monitors of HV and current to the APAs
\item Survey of wire positions
\item Temperature sensors {\it in situ} at several places in the volume
\item Purity monitors at several positions within the volume
\item Front-end electronics response calibration pulsers.
\end{itemize}
	These systems provide the initial input to the model for electric
field, and for calculations of average drift velocity, electron lifetime, and
electron diffusion.  The goal of the calibration systems we describe in this
note is first to measure the response parameters with granularity in time and
position that cannot be accomplished by the above systems, and to test the
resultant model with tracks of known positions, trajectories, and energy
deposits.
	
%%%%%%%%%%%%%%%%%%%%%%%%%%
\subsection{Laser System}	
%Much of this comes from J. Maricic's laser system proposal from Hawaii

	The primary goal of the laser system is to provide realtime
measurements of the parameters that determine the detector response: electron
diffusion, and position-dependent variations in electron lifetime and drift
velocity.  One way of viewing this is that the laser system provides a
map of the electric field and impurities in the detector---in fact, it is
really a generator of position-dependent corrections to the field and
impurities.  As a realtime device, it also provides measurements of how these
things vary with time.

	Given the high space charge that will build up due to the high cosmic
flux on the surface, we expect the ProtoDUNE-SP detector to have large
variations in electric field as a function of position.  Simulation of an
example uniformly distributed space charge in protoDUNE using the simulation
package SpaCE~\cite{mooney} shows great variations in all three components of
the electric field $E_x$, $E_y$ and $E_z$ (up to 12\%), leading to a 4.5!cm
longitudinal and 20~cm transverse track distortion at the nominal electric
field of 500 V/cm and protoDUNE nominal geometry with 3.6m drift length.
Measuring the drift velocity from known laser trajectories allows a mapping of
that field.

	To accomplish this mapping, the laser system needs to be able to
illuminate a large fraction of the TPC volume, and the trajectory of the laser
needs to be measured by detecting the light on the far side of the active
volume.

%%%%%%%%%%%%%%%
\subsubsection{Laser System Design}

	The design of the laser system follows that of a system built for the
MiniCAPTAIN detector by the University of Hawaii.  In the current ProtoDUNE-SP
cryostat design includes four entry ports for laser periscopes, two in each of
the two volumes between each of the anode planes and the single cathode plane.
Two periscope entries are important to increase the volume of the TPC that will
be illuninated and to ensure laser access even in a case of shortening the
drift distane from 3.6~m to 2.5~m (or even 2.0m which is still under
discussion). The final choice will depend on how well the field distortions can
be handled. Each periscopes will be placed behind the TPC cage and reflects the
laser beam from a mirror at the bottom of the periscope into the TPC volume
through the gap between anode assembly planes. Entire periscopes can rotate
around their central axis for azimuthal coverage. The mirror at the bottom of a
periscope can swivel in the vertical plane to send the beam at different polar
angles.  
%A conceptual drawing of the cones of light

%defing the coverage of the detectors from two laser beams is shown in the

%picture Fig. 8. 
%In the design, the laser beam will enter the TPC volume through
%the gaps between two adjucent eld cage frames.  Field cage is made out of
%hemispherical metal pipes held together in the fence like frames made out of
%FR-4. There may be changes in the nal eld cage design where each side will
%consist of 2x4 frames, instead of 2x3 as shown in the gure.  Consequently,
%this will change the location of the bottom mirror and the overall height of
%the periscope.  Fig. 10 shows details of the gap size between the frames. At
%the locations of the bottom

Laser beam pulses will be generated by a pulsed Nd:Yag laser's fourth harmonic
at 266~nm. The laser must deliver tens of mJ per pulse at 266 nm. The pulse
frequency is typically 10 Hz and pulse duration around 5 ns. Important
characteristic of the laser is small divergence of the beam. Two companies have
been identifid: Quantel and Surelite that feature very low divergence beams at
less than 0.5 mrad. This is important feature will ensure that the beam remains
relatively tightly collimated over long distances, retaining high ux and
producing a narrow ionization track inside the TPC. Laser beam will be directed
through a set of mirrors to the mirror at the top of the cryostat oriented at
\SI{45}{\degree} angle to the beam, that bends the beam at right angle into the periscope,
hitting the mirror at the bottom of the periscope and delivering the beam into
the TPC.

While the direction of the laser beam will be very well known based on the
reading from a rotary position encoder and linear actuator that is part of the
periscope assembly, there will still be some residual uncertainty or
unpredictable shift in the pointing direction.  Having in mind long length of
the ionization track of more than 7 m, even a small offset in the pointing
direction can lead to vastly different ionization track location, especially
close to the end of the track. Such inaccuracies will directly impact the
ability to precisely calibrate any variations in the electric drift field.

To mitigate this issue, each laser pulse will be localized by a laser
positioning system (LPS) that will use groups of pin dodes.  Five groups of
these diodes are distributed in a cross-like pattern.  One group is directly
opposite the mirror when it is facing the front at \SI{45}{\degree} tilt mirror.  Two
groups are in the same vertical plane close to the top and bottom of the same
vertical axis. The two final groups are to the left and to the right from the
central group. The electronics used to collect signals from the LPS will be
commercially made by CAEN. 
%We have received the quote for the V1740D high
%density modules that take 64 channels each. 
%There are several unoccupied ports
%on the opposite side from the laser ports that can be


%%%%%%%%%%%%%%%%%%%%%%%%%%
\subsection{Cosmic Tracker}

	While there may be motivations from a reconstruction standpoint for
cosmic veto, from a calibration standpoint what we are interested in a tracking
system to provide known cosmic trajectories.  The tracker may also be used for
off-beam triggering on stopped muons, as well as detector triggering during
times when there is no beam.  

	Counters for a cosmic tracker are available from spares made for the
Double CHOOZ experiment by the Universit of Chicago. Light created in the
counters is transmitted via fibers to a 64-anode PMT and readout via
electronics that use a USB interface.  The counter size is roughly 2.5~cm, but
multiple layers of the counters could be used for finer granularity.

	Although there are many possible configurations for deployment of the
counters, a particularly interesting deployment would be fore-and-aft,
therefore triggering on muons that had trajectories similar to beam events. In
this configuration, the counters could also serve as a tagger for beam halo
events.  By tagging such events, only those whose energy and position are
reasonably well constrained will be part of the beam calibration.

	While this deployment scheme works well for calibration plans, it
provides little help on tagging cosmics that could cause difficulty for
reconstruction algorithms, since the vast majority of those will be
downward-going.  

	There are two primary goasl for the cosmic tracker. The first is to use
the parameters measured by the purity monitors and temperature sensors, and the
measurements of electric field as a function of position and time provided by
the laser system, to predict the precision and bias of reconstruction of
straight tracks from cosmic muons. The reconstruction of the tracked cosmics
can then be compared to the results of reconconstructed simulated events. If
the differences are within the uncertainties derived from the initial parameter
measurements, then we have confidence that the Monte Carlo model is predictive
with the precision needed to measure beam event responses.  

	The second goal is to provide a measurement of the detector's energy
scale---essentially the global mapping of deposited energy to ADC counts.  Such
a measurement can use either the through-going cosmic rays that hit both the
fore and aft counters, corrected for the expected energy spectrum (and
calculated dE/dx for the primarily minimum-ionizing cosmics), to provide a
MIP-based energy scale, or to trigger on stopped cosmics outside of the beam
gate, and using the part of the muon trajectory that has a residual range
anticipated to be in the minimum-ionizing regime (similar to what was done for
MINOS).  
	

%%%%%%%%%%%%%%%%%%%%%%%%%%
\subsection{PDS Calibration} -- ANL -- Zelimir
     Mechanical design -- Zelimir interface with Bob Sands?

%%%%%%%%%%%%%%%%%%%%%%%%%%
\subsubsection{Run Plan}
     Run/Measurement plan  -- Toups ?
     Light yield vs. field
     timing studies,
     beam off trigger

%%%%%%%%%%%%%%%%%%%%%%%%%
\subsection{QC Procedures}

