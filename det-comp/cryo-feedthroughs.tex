%\chapter{det-comp}


%%%%%%%%%%%%%%%%%%%%%%%%%%%%%%%%%%%%%%%%%%%%%%
%\section{Anode Plane Assemblies}

%%%%%%%%%%%%%%%%%%%%%%%%%%%%%%%%%%%%%%%%%%%%%%
%\section{Cathode Plane Assemblies}

%%%%%%%%%%%%%%%%%%%%%%%%%%%%%%%%%%%%%%%%%%%%%%
%\section{Field Cage}

%%%%%%%%%%%%%%%%%%%%%%%%%%%%%%%%%%%%%%%%%%%%%%
%\section{HV components}

%%%%%%%%%%%%%%%%%%%%%%%%%%%%%%%%%%%%%%%%%%%%%%
%\section{TPC front-end electronics and DAQ}

%%%%%%%%%%%%%%%%%%%%%%%%%%%%%%%%%%%%%%%%%%%%%%
%\section{PDS front-end electronics and DAQ}

%%%%%%%%%%%%%%%%%%%%%%%%%%%%%%%%%%%%%%%%%%%%%%
\section{Cryostat and feedthroughs}

The cryostat consists of a steel warm outer structure, layers of insulation and an inner cold membrane.  The outer %steel warm 
structure (shown in Figure~\ref{fig:warm-vessel-layout}), which provides %represents 
the mechanical support for the  %inner membrane cold cryostat 
membrane and its insulation,  consists of vertical beams that alternate with a web of metal frames. It is constructed to %capable of 
withstand the hydrostatic pressure of the liquid argon, the pressure of the gas volumes and %all possible 
satisfies the external constraints. %The steel warm structure for NP04 is shown in Figure~\ref{fig:warm-vessel-layout}.
In particular, this structure is to be constructed in EHN1 without any mechanical attachment to the floor or the building side walls.  \fixme{what are the ext constraints? Also, the text on the figure is too small; is it intended to indicate what each color is? Can you clarify and make sure each part/color is labeled?}
%
\begin{cdrfigure}[Warm vessel layout]{warm-vessel-layout}{lWarm vessel layout showing the various major components}
  \includegraphics[width=0.9\textwidth]{warm-vessel-layout}
\end{cdrfigure}
%
%The main requirement is that this mechanical structure is constructed in EHN1 without any mechanical attachment to the floor or the building side walls.  Inside the steel structure, a 10 mm skin of stainless steel plates are welded to provide a gas barrier to the outside.
Inside the steel structure, a 10-mm thick skin of stainless steel plates will be welded to provide a gas barrier to the outside.
The top of the cryostat will be accessible for installation of the detector elements, the electrical/signal feedthrough, the detector supports and other cryogenics services.  Figure~\ref{fig:cryo-structure-2d} is a mechanical drawing of the structure. The dimensions %requirements 
are dictated by the required %need to provide a sufficient 
active volume of LAr:  % to the detector. This means a transversal internal dimension of the liquid volume of 
width = 8.548~m, length = 8.548~mm and height = 7.900~m. The dimensions \fixme{of the active volume or total volume (which would equal the entire volume inside the membrane)?} %have been adapted in order to 
ensure that all crossing penetrations are arranged as requested and that there is enough space for maintenance. \fixme{Does this mean: the dimensions of the active LAr volume take into account the volume taken up by penetrations in the roof? Are pipes and cables and such allowed to penetrate the active volume? I assume not the fiducial volume...?} 

\begin{cdrfigure}[2D mechanical drawing of the cryostat outer structure]{cryo-structure-2d}{2D mechanical drawing of the cryostat outer structure}
  \includegraphics[width=0.95\textwidth]{cryo-structure-2d}
\end{cdrfigure}
\fixme{The elements in this figure are pretty small; I would individually add in only the portions you call attention to, and allow them to be larger.}

Section view B in Figure~\ref{fig:cryo-structure-2d} shows a detail of the warm structure on the outside, 10 mm SS skin, 800 mm of insulation and the inner cold vessel. 
\fixme{I would argue that it doesn't show it very clearly}
A secondary membrane that is located within the insulation layer is not shown in this view.  The cold vessel is based on the GTT membrane technology.  \fixme{add ref} The initial thermal requirements 
\fixme{why initial?} call for %an insulation thickness of 800 mm, including the primary and secondary membrane. 
a thickness of 800\,mm, including the insulation, and the primary and secondary membranes. These two membranes provide a first and second level of containment. There is no requirement at this point for additional containment at the level of the warm steel structure. The SS skin of 10-mm thickness between the warm structure and the insulation will provide an effective gas enclosure, which will allow control of the argon atmosphere inside the insulation volume.
All necessary information can be found in \url{https://edms.cern.ch/document/1531438/3} \fixme{make it a citation}
The 3D detailed cad model is visible in \url{https://edms.cern.ch/document/1531439/3} \fixme{make it a citation}

Prior to installation of the GTT insulation and %cold liquid 
inner membranes, the gas tightness of the SS 10-mm membrane will be qualified \fixme{verified?} by CERN using dye penetrant analysis, local vacuum-bag techniques, and He leak %leaks sniffing 
detection at the level of the natural He present in the atmosphere ($\sim2-3 \times 10_{-6}6 mbar/l/sec$). A report will be presented to GTT.

%%%%%%%%%%%%%%%%%%%%%%%%%
\subsection{Storage characteristics}


%The expected function of the cryostat is to store LAr in liquid form at atmospheric pressure or just above. 
The cryostat will be placed in an existing CERN experimental hall at CERN (EHN1 building). The 
cryostat is required to store LAr at a temperature between 86.7\,K and 87.7\,K with a pressure inside the 
tank of 950 < P < 1100\,mbar. %A special emphasis has been made on the thermal fluxes. 
The thermal fluxes must be tightly controlled, i.e., %They must be controlled and 
they will be kept under 5\,W/m2 on %the walls/floor 
the inner membrane that is in contact with liquid, in order to prevent boiling of the LAr.

The storage parameters of the cryostat are as follows:
\fixme{I changed this from enumerate to itemize; this isn't really an ordered list.}
\begin{itemize} % Need  \usepackage{enumerate} to append this: [(a)]
\item The inner dimensions are 7900\,mm high $\times$ 8548\,mm length $\times$ 8548\,mm wide.  This corresponds to a total volume of $\sim$580 m$^3$. 
\item Tank liquid capacity (assuming a $\sim$4\% ullage): $\sim$ 557 m$^3$
\item Residual Heat Input (RHI): 5-6\,W/m$^2$
\item Insulation weight: 90 kg/m$^3$  
\item Insulation thickness (all included): 0.8\,m 
\item Design pressure: Max 1350 mBar / Min 950 mBar.  The 1350 mBar is for an accidental condition during the cryogenics operation.
\item Operating temperature: 86K-89\,K
\end{itemize}

Figure~\ref{fig:cryo-overall-dim} shows the inner dimensions of the cryostat, dimensions of the \SI{10}{mm} SS skin and the overall outer dimensions of the warm structure.
\fixme{It only shows in 2D}

\begin{cdrfigure}[Cryostat overall dimensions]{cryo-overall-dim}{Cryostat overall dimensions}
  \includegraphics[width=0.8\textwidth]{cryo-overall-dim}
\end{cdrfigure}

%%%%%%%%%%%%%%%%%%%%%%%%%
%\subsection{Fe warm cryostat design}
\subsection{Warm Fe cryostat design}

\fixme{Why does title say Fe for iron if it's stainless steel?}

The design and structural analysis of the warm support structure including the 10-mm SS gas containment membrane is not a part of this study and is %has to be 
treated as a CERN deliverable. \fixme{What is its status?}

\fixme{Anne stopped here 8/19}

The structural beams are the main elements of the floor, roof and side walls of the warm structure and the corner structural elements can be found only at the corners where the side wall is connected with the other side wall. The structural beams are the principal load bearing elements of the structure. Their purpose is to withstand the hydrostatic pressure from the LAr, as well as to support the roof load. They consist of IPE V 600 standard profile. 
\fixme{add and reference figures   Figure~\ref{fig:struct-beam-detail} and Figure~\ref{fig:corner-struct-element-detail} }

\begin{cdrfigure}[Structural beam detail]{struct-beam-detail}{Structural beam detail}
  \includegraphics[width=0.4\textwidth]{struct-beam-detail}
\end{cdrfigure}

\begin{cdrfigure}[Corner structural element detail]{corner-struct-element-detail}{Corner structural element detail}
  \includegraphics[width=0.4\textwidth]{corner-struct-element-detail}
\end{cdrfigure}


%%%%%%%%%%%%%%%%%%%%%%%%%
\subsection{The web interlink structure}

The web interlink structure purpose is to provide an adequate support of the polyurethane insulation of the membrane. They consist of IPE O 270 profiles. \fixme{Reference Figure~\ref{fig:web-interlinks-detail}}

\begin{cdrfigure}[Web interlinks detail]{web-interlinks-detail}{Web interlinks detail}
  \includegraphics[width=0.4\textwidth]{web-interlinks-detail}
\end{cdrfigure}


Additional \SI{10}{mm} stainless steel plates are welded to the web interlink structure to create a gas containment barrier to the inside, as well as to ensure even better support for the membrane insulation. 
The web interlink structure is also used for the floor, in-between the corner beams, as well as for the roof structure.

A very detailed technical report is available at \url{https://edms.cern.ch/document/1531442/1} \fixme{make it a citation}.
In this report the following items are discussed:
\begin{itemize}
\item all of the various load conditions including boundary conditions and the safety factors,
\item buckling analysis,
\item design and stress calculations for all of the bolted connections,
\item finite element analysis calculations for the 10 mm skin and structure, and
\item the compliance with the required codes.
\end{itemize}

%%%%%%%%%%%%%%%%%%%%%%%%%
\subsection{Cold GTT Vessel}

Inside the warm support structure, which includes the stainless steel gas enclosure membrane, the GTT cold vessel will be installed. It consists of a thermal insulation, a primary corrugated stainless steel membrane, as well as a secondary thin membrane, to provide primary and secondary liquid containment. A cross sectional view of this is shown in Figure~\ref{fig:xsec-insulation-layers}.

\begin{cdrfigure}[Cross section of insulation layers on membranes]{xsec-insulation-layers}{Cross section of insulation layers on membranes}
  \includegraphics[width=0.8\textwidth]{xsec-insulation-layers}
\end{cdrfigure}


The primary membrane is made of corrugated stainless steel 304 L and is 1.2 mm in thickness.  The standard size of the sheets is 3 m x 1 m.  The secondary membrane is made of Triplex.  This is a composite laminated material of a thin sheet of aluminium between two layers of glass cloth and resin.  It is positioned inside the prefabricated insulation panels between two of the insulation layers.  Details of the two membranes are shown in Figure~\ref{fig:prim-2nd-membranes}.   The insulation is made from reinforced polyurethane foam.  The insulation panels are bonded to the inner 10 mm skin using resin ropes.  The insulation layers will be instrumented with gas inlets, outlets, temperature and pressure sensors.

\begin{cdrfigure}[Primary membrane, prefabricated insulation and secondary membrane]{prim-2nd-membranes}{lLeft shows the primary stainless steel corrugated membrane and right shows the prefabricated panels of insulation and the secondary membrane}
  \includegraphics[width=0.45\textwidth]{primary-membrane}
  \includegraphics[width=0.45\textwidth]{prefab-insul-2nd-membrane}
\end{cdrfigure}

%%%%%%%%%%%%%%%%%%%%%%%%%
\subsection{TCO (Temporary Construction Opening)}

A dedicated access window will be necessary to install the NP04 detector.  This is referred to as the TCO.  This means that no insulation of membrane can be installed at the beginning in this location. Once the detector installation has progressed as far as possible and all of the large TPC components are inside the cryostat, the TCO will be closed.  The 10 mm SS skin, insulation and cold membranes will be installed and welded in place. \fixme{reference Figure~\ref{fig:front-tco-np04}}

\begin{cdrfigure}[Front view of the TCO for the NP04 cryostat]{front-tco-np04}{Front view of the TCO for the NP04 cryostat}
  \includegraphics[width=0.8\textwidth]{front-tco-np04}
\end{cdrfigure}


%%%%%%%%%%%%%%%%%%%%%%%%%
\subsection{LAr pump penetration}

On one side wall, as low as possible, a special penetration has to be foreseen to connect the extraction liquid argon pumps that will allow the liquid recirculation to an external filtering system. For both solutions the penetration will be the same but placed in different locations (see section 8.2).
To keep the high level of purity required, an external pump is connected on the one side to the bottom of the liquid, through a dedicated system of safety valves. This penetration requires a local modification of the insulation panels and the SS primary membrane, and will be a crossing tube with diameter of 168 mm for the insulation and the membrane and a larger diameter hole at the stainless steel plate.

%%%%%%%%%%%%%%%%%%%%%%%%%
\subsection{Beam window penetration}

\fixme{Paola needs to update this section to describe the CERN beam penetration design}
Once constructed, the NP04 detector will be exposed to the charged particle beams from the SPS accelerator. To minimize energy loss and multiple scattering of the beam particles in the dead material of the cryostat and its insulation, a beam window will be inserted.  The beam window dimension is about 220 mm in diameter. The outer portion of the beam window penetration is a vacuum pipe that extends from the H4 beam line through the outer insualtion layer and ends at the secondary membrane. The portion of the foam insulation between the secondary and the primary membrane is replaced with a lower density foam (xyz kg/cm$^3$). To maintain structural integrity, the plywood supporting the primary membrane in the vicinity of the beam window penetration is replaced with a nomex honeycomb plate. In this design, both of the primary and secondary stainless steel membranes remain intact. 


%%%%%%%%%%%%%%%%%%%%%%%%%
\subsection{Roof signal, services and supports penetrations}

The penetrations through the cryostat have been arranged by positions and diameters (see Section 8). Most of the penetrations are placed on the ceiling of the cryostat. They have been differentiated into two main groups according to their function and the thermal stresses they will be submitted to: whether they can be used to support the weight of the detector or not.
The penetrations on the roof of the cryostat for the NP04 cryostat details are shown below.  A 3D cad model to identify all position can be found at:
\url{https://edms.cern.ch/document/1543241/3} and drawing CENDUNCR0016. \fixme{make them citations}

\fixme{add reference to Table~\ref{tab:roofpenetrations} }

\begin{cdrtable}[Detector penetrations, roof]{llll}{roofpenetrations}{Detector penetrations, roof}
Component & Header Column 2??  & Quantity & Value \\ \toprowrule
West TPC translation suspension: &  N.3 & crossing tube diameter & \SI{200}{mm}\\ \colhline
Center TPC translation suspension:& N.3 & crossing tube diameter & \SI{200}{mm}\\ \colhline
East TPC translation suspension:& N.3 & crossing tube diameter & \SI{200}{mm}\\ \colhline
Signal cable chimney FTs:&  N.8& crossing tube diameter & \SI{250}{mm}\\ \colhline
Spare on Signal cable row FTs: & N.2 & crossing tube diameter & \SI{250}{mm}\\ \colhline
Laser FTs: & N.1 & crossing tube diameter & \SI{160}{mm}\\ \colhline
Calibration Fiber CPA FT:& N.1 & crossing tube diameter & \SI{250}{mm}\\ \colhline
Spare on CPA line FTs:& N.2 & crossing tube diameter & \SI{150}{mm}\\ \colhline
HV FT: & N.1 & crossing tube diameter & \SI{250}{mm}\\ \colhline
Manhole:&  & crossing tube diameter & \SI{710}{mm}\\ \colhline
Angled beam windows -- west side:&  & crossing tube diameter & \SI{250}{mm}\\ \colhline
 &  &  Vertical: \SI{11.342}{\degree} & \\ \colhline
 &  &  Horizontal: \SI{11.844}{\degree} & \\ \colhline
TCO - side:&  & \num{1200}\si{mm} $\times$ \num{7300}\si{mm} & \\ \colhline
 Cryogenic pipes - roof: &  &  & \\ \colhline
 &  N.4& crossing tube diameter & \SI{250}{mm}\\ \colhline
 &  N.1& crossing tube diameter & \SI{304}{mm}\\ \colhline
 & N.5 & crossing tube diameter & \SI{152}{mm}\\ \colhline
 &  N.1& crossing tube diameter & \SI{125}{mm}\\ \colhline
 &  N.3& crossing tube diameter & \SI{250}{mm}\\ \colhline
 Cryogenic pipes -- north side: &  N.1& crossing tube diameter & \SI{168}{mm}\\ 
\end{cdrtable}

%%%%%%%%%%%%%%%%%%%%%%%%%
\subsection{Mechanical mounts and supports}
\begin{itemize}
\item Introduction 
\item Layout and general design
\item Technical information and specifications
\item Interface with cryostat and warm structure
\item Interface with TPC including electrical/grounding 
\item Structural analysis and factors of safety
\end{itemize}

%%%%%%%%%%%%%%%%%%%%%%%%%
\subsection{QC Procedures}

