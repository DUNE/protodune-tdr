%\chapter{det-comp}


%%%%%%%%%%%%%%%%%%%%%%%%%%%%%%%%%%%%%%%%%%%%%%
%\section{Anode Plane Assemblies}

%%%%%%%%%%%%%%%%%%%%%%%%%%%%%%%%%%%%%%%%%%%%%%
%\section{Cathode Plane Assemblies}

%%%%%%%%%%%%%%%%%%%%%%%%%%%%%%%%%%%%%%%%%%%%%%
\section{Field Cage (FC)}
\label{detcompsec-fc}
%%%%%%%%%%%%%%%%%%%%%%%%%
%\subsection{Scope, Requirements and Design Parameters}
\subsection{Overview and Requirements} 

In the TPC, each pair of facing cathode and anode planes forms an electron-drift region. A field
cage (FC) must completely surround the four open sides of this region to provide the necessary boundary
conditions to ensure a uniform electric field within, unaffected by the presence of the cryostat walls.

The FCs are constructed using multiple copper-clad FR-4 sheets reinforced with fiber
glass I-beams to form modules of 2.3 m $\times$ 3.6 m in size. Parallel copper strips are etched on
the FR-4 sheets using standard printed circuit board fabrication techniques. Strips are biased at
appropriate voltages provided by a resistive-divider network. These strips create a linear electric potential
gradient in the LAr, ensuring a uniform drift field in the TPC active volume. Simulations
have shown that the non-uniformity of the drift field quickly drops to about 1\%, roughly a strip
pitch away from the field-cage surface.
\fixme{The above taken from sec 4.3.4 of the detector volume of the CDR as an intro. May need update for protodune-sp?? Anne}

The FC is required to:
\begin{itemize}
\item provide the nominal drift field of 500V/cm;
\item withstand $-$180kV near the cathode;
\item define the drift distance between the APAs and CPAs to <1~cm;
\item limit the electric field in the LAr volume to under 30~kV/cm;
\item miminize the peak energy transfer in case of a HV discharge anywhere on the field cage or cathode;
\item provide redundancy in the resistor divider; \fixme{this feels incomplete... ``in the resistor-divider chain?}
\item maintain the divider current much greater than the ionization current in the TPC drift cell, yet less than the power supply current limit when all dividers are connected in parallel;
\item be modular in form such that they can be easily installed in the cryostat;
\item provide support for the beam plug;
\item allow calibration laser beams to enter into the active volume; 
\item support a 200-lb. person standing on the support beam of the bottom field cage module;
\item be configurable to either 3.6~m or 2.5~m drift length inside the cryostat; and
\item prevent any trapped volume.
\end{itemize}

%%%%%%%%%%%%%%%%%%%%%%%%%
\subsection{Mechanical design}

The ProtoDUNE-SP TPC will have six top and six bottom FC assemblies, arranged three across each horizontal edge of the two drift regions. It will have 
four end-wall panels, one at each vertical edge of the two drift regions, see Figure~\ref{fig:fc-overview} and~\ref{fig:fc-endwall_module}.
Each endwall panel consists of four assemblies in ``landscape'' orientation, stacked vertically.
FC assemblies are constructed from pultruded G10 I-beams and box beams that support extruded field-shaping aluminum profiles. The support structure for each of the top and bottom FC assemblies consists of two main I-beams that are 3.6~m long, and three cross I-beams that brace the main I-beams for structural stability.
\fixme{One field cage with many assemblies or modules? Clarify assembly vs module} 


\begin{cdrfigure}[An end wall field cage module]{fc-endwall_module}{A view of an end wall field cage module}
\includegraphics[width=0.8\linewidth]{tpc_fc_endwall_module.png}
\end{cdrfigure}

The main I-beams have cutouts to hold the field-shaping profiles. Main I-beams are spliced at 2.5~m to facilitate drift distance  
of 2.5~m. Splice joints and cross I-beam joints are held together using an arrangement of shear pins and plates. 

Aside from the profiles themselves, the nuts and bolts holding them, and the ground planes, all FC components are made of insulating material. The material selected for these structural components is fiberglass-reinforced plastic (FRP), which will prevent binding when the structure is at cryogenic temperatures. The ground planes are made of stainless steel. 

\fixme{from docdb 1504 sec 6.3 I can't tell if there's ONE ground plane at the TOP, ONE at the BOTTOM or two. In this section it appears plural. 1504 needs clarification. }

The inward-facing face of the ground planes will be approximately 20~cm away from the top of the field-shaping profiles. The ground planes are mounted at a fixed distance from the field shaping profiles by standoffs, as shown in Figure~\ref{fig:fc-with-ground-planes}, which shows ground planes over I-beams and cross beams.

\begin{cdrfigure}[The field cage with ground planes]{fc-with-ground-planes}{The field cage with ground planes}
\includegraphics[width=0.8\linewidth]{fc-with-ground-planes}
\end{cdrfigure}

The parallel metal profiles in each FC assembly 
 are interconnected by a resistive divider chain, and supported by the FRP beams that span the drift distance.  Between adjacent field cage assemblies, however,  
the metal profiles are neither mechanically nor electrically connected. Gaps between assemblies range from a few millimeters to a few centimeters are designed into the TPC assembly to ensure sufficient clearance for the installation.  The electrical isolation between the field cage modules minimizes the peak energy dump in case of a HV discharge.


%%%%%%%%%%%%%%%%%%%%%%%%%
\subsection{Electrical design}

Given a large standoff distance between the FC and the grounded cryostat wall, it is relatively easy to design a FC that meets the 30-kV/cm E field limit with 180-kV bias.  However, It becomes challenging to reduce the clearance between the FC and ground in order to make more efficient use of liquid argon.  This requires an electrode with a low profile, rounded edges, no trapped volume, and low cost.  Several commercially available roll-formed metal profiles were studied and appear to meet these requirements. \fixme{what's challenging about it then?}

Figure~\ref{fig:fc-schematic} is a schematic of the electrical design of  the CPA and a top/bottom field cage module pair.

\begin{cdrfigure}[Field cage schematic diagram]{fc-schematic}{A schematic diagram of the CPA and a top/bottom field cage module pair}
\includegraphics[width=0.8\linewidth]{tpc_fc_schematic.png}
\end{cdrfigure}

%%%%%%%%%%%%
\subsubsection{Electrostatic analysis}

The Dahlstrom Roll Form \#1071 was found to have the lowest surface E field, which was about 12~kV/cm when biased at 180~kV with only a 20-cm ground clearance (see Figure~\ref{fig:fc-profile1071}).

\begin{cdrfigure}[2D FEA of roll formed profiles]{fc-profile1071}{A 2D FEA of a configuration using profile 1071, and a conceptual design of the field cage module}
\includegraphics[width=0.8\linewidth]{tpc_fc_profile1071.png}
\end{cdrfigure}
  
In order to maintain a modular design of the field cage while minimizing peak energy transfer in a discharge, the field cage will be constructed with electrical isolation between neighboring modules. If a discharge occurs on one field cage, the electrodes from the neighboring modules will not arc over and cause a domino effect.  This requires a 
electrical insulation between profile ends of the order of 180~kV.  UHMW PE caps of 5-mm thickness are placed over both ends of each profile to serve this purpose. This technique also limits the exposed electric field in LAr at the corner of the field cage, see Figure~\ref{fig:fc_corner}. 

\begin{cdrfigure}[3D FEA of field cage corner]{fc_profile_corner}{A 3D FEA of a field cage corner.  The PE caps (in outline form) limit the exposed E field below the 30kV/cm threshold.}
\includegraphics[width=0.6\linewidth]{tpc_fc_profile_corner.png}
\end{cdrfigure}

The center-to-center distance between the profiles is set to 6~cm, and the inner surface of all profiles on a field cage module is placed 5~cm beyond the nearest surface of the TPC active volume (defined by the APA active aperture over the drift distance). The E field uniformity at the edge of the active volume is expected to be within $\pm$2\% of the nominal value.


%%%%%%%%%%%%  
\subsubsection{Surge suppressor on FC}

\fixme{Not clear what design will be used; discusses alternatives}

The resistors along the divider, Figure~\ref{fig:fc-divider-view}, provide a linear DC voltage gradient. However, at shorter time scales ($\ll$1~s), the electrical behavior of the divider is determined by the various capacitances on and between each electrode, and  is no longer linear at this time scale. 

\begin{cdrfigure}[The resistive divider]{fc-divider-view}{A view of a resistive divider}
\includegraphics[width=0.8\linewidth]{tpc_fc_r_divider.png}
\end{cdrfigure}

A perfect capacitive divider requires the capacitance of each node to ground to be zero.  In reality, there is always a finite capacitance from each node to ground, and these capacitances resist change in the voltages on the nodes. In the event of a HV breakdown between the cathode and ground (cryostat), the cathode voltage quickly collapses to ground, but the first FC electrode-to-ground capacitance keeps its voltage from changing instantaneously to follow the cathode voltage, resulting in a momentarily larger voltage differential between the cathode and the first FC electrode. This voltage differential can be a significant fraction of the cathode operating voltage, large enough to cause HV breakdown between the two electrodes, or worse yet, to destroy the divider resistors between the two electrodes.

A natural solution to this problem is to significantly increase the capacitance between the nodes of this divider. This was the approach adopted in the 35-ton prototype's field cage through the use of double-sided printed circuit boards (PCB).  However, the cost of the PCB version of the field cage at DUNE scale is very high, and adding discrete HV capactitors between divider taps is also expensitive.

An alternative is to use surge surpressors in parallel with the divider resistors to divert the transient current from the resistors. Extensive tests have been done by MicroBooNE (docdb 3242, arXiv:1406.5216v2) \fixme{add in bib and cite here} on the use of metal oxide varistors (MOVs) and gas discharge tubes (GDTs) as a means of limiting the over-voltage condition in the event of a HV discharge in the TPC. 

Both types will work for the purpose of restricting the voltage differential between field cage rings in LAr temperature. 
A GDT quickly shorts the terminals when the voltage differential exceeds a threshold, while
a varistor changes its resistance to keep the voltage differential near the threshold voltage.
The smooth transition and well defined clamping voltage of the varistors are preferred to the abrupt switching of the GDTs.
The varistors could also function as redundant ``resistors'' in a divider chain in case a resistor is open circuit. \fixme{open circuited?}

One readily available MOV famility \fixme{family?} with high threshold voltage is the Panasonic ERZ-VXXD182.  These have a threshold voltage around 1600~V.  Two of these in series could work with the current 3-kV differential between divider taps.  However, this configuration would not allow raising the E field much above the nominal 500~V/cm.  To allow some headroom in operating field, three such MOVs in series would be needed.

%%%%%%%%%%%%%%%%%%%%%%%%%
%\subsection{Validation tests of roll-formed FC design}


\begin{cdrfigure}[The field cage test setup]{fc-test}{The field cage test setup. 
 {\bf Left:} schematic drawings of the cage showing the main elements: metal profiles, I-beams, ground planes.
  {\bf Right:} Picture of the realized setup.}
\includegraphics[width=0.45\linewidth]{tpc_fc-test-1.png}
\includegraphics[width=0.45\linewidth]{tpc_fc-test-2.png}
\end{cdrfigure}
\fixme{Remove Figure~\ref{fig:fc-test}? Not shown as ``remove it'' in Gina's printout}


%%%%%%%%%%%%%%%%%%%%%%%%%
\subsection{Ground plane (GP) design}
%% by A. Zani

In order to confine the electric field in the liquid argon region, it is foreseen to install a grounded metallic plane,  between the upper field cage module and the liquid-gas interface. The design of such a Ground Plane (GP) \fixme{need to define acronym above and just use it} is inspired by the one from the ICARUS T600 detector, and it is meant to limit the residual electric field in the liquid below the usual 30~kV/cm value. 
\fixme{Is it to keep the E field from going outside the LAr volume or is it to keep the residual field that's already in the LAr volume very small?}
%The design details of the planes were verified to comply with the requests on residual electric field with FEA.  It is noted that a similar GP could be added in front of all the other Field Cage (FC) modules, in order to smooth the field in the LAr dead volume. However, so far it is foreseen to add an actual GP only below the bottom FC, to further smooth the field in the region where pipings for the cryostat filling are running. The distance between the cryostat walls and the end-wall field cage does not require to insert a GP, instead.
The design details of the planes were verified for compliance with the requests on residual electric field with FEA. \fixme{I cannot parse prev sentence} 

In the ProtoDUNE-SP configuration, the GP will be put at a distance of 200~mm from the FC profile, with a structure of $6$ mm holes, $10$ mm pitch ($\sim 25\%$ transparency): the lower fraction of pierced surface is verified by simulations to maintain the field within the required values. The edges of the plates, $20$ mm high, are rounded at $5$ mm, while the holes rounding radius, at production is around $0.5$ mm. The liquid level is expected to be at 40 mm above the GP bottom, i.e. 20 mm above the edges. The radius of curvature for the holes is not a strict requirement. It depends on the punching technique, and usually is at around $0.5$ mm. The actual requirement is to have the hole curvature on the inside (of the TPC) looking out.
\fixme{The previous pgraph needs clarification}

Two sets of pieces \fixme{two sets of GPs? Define ``piece''} were initially produced in Europe:
\begin{itemize}
\item 8 pieces of dimension $198 \times 571$ mm (weight $< 1$ kg each) to be installed in the CERN field cage prototype,
\item 6 more pieces of  $525 \times 2318$ mm (weight around $8.5$ kg each), which represent full scale components for ProtoDUNE-SP. %The drawing of this second set of pieces, sent to the U.S. for test assemblies, in shown in Figure~\ref{fig:gp_panels}. -- this figure was removed per Gina
\end{itemize}



%%%%%%%%%%%%%%%%%%%%%%%%%
\subsection{FC and GP modules}

%The design details of the FC+GP modules for ProtoDUNE is described and shown in the next section. 
One field cage and ground plane (FC+GP) module will be made of six pieces, aligned 
along their long (2318-mm) dimension 
 in order to match the APA and FC module widths. The planes are connected to the FC beam with additional 
G10 pieces that are also used to connect adjoining 
GP panels. \fixme{piece vs panel?}

Figure~\ref{fig:fc_full} shows a 3D model of one fully assembled FC+GP module.

\begin{cdrfigure}[CERN Prototype]{fc_full}{3D model of one fully assembled FC+GP module, for the top of the field cage. }
\includegraphics[width=0.8\linewidth]{tpc_topFC.png}
\end{cdrfigure}

The electrical continuity between consecutive panels can be made 
with metallic screws (with holes on the planes edges) or with looser connections, e.g., copper strips, that better adapt to the shrinking of the structure during cooldown. 
As for most detector systems, the GP will %should 
be referenced to the detector ground, located 
at the cryostat top.

The GP modules \fixme{modules or panels?} will be installed on their corresponding FC modules in the clean room outside the cryostat, to facilitate the connections. The description of how the top/bottom FC modules are assembled and connected to the CPA before insertion in the cryostat is provided in Section  \fixme{need to find and add reference}.

Further GP panels need to be attached to the top FC module:
\begin{itemize}
\item Smaller panels will have to be connected on the modules on one side of the CPA so that, once in position, they 
cover the CPA frame. Their dimensions are 
still to be defined, depending on the final design of the CPA hanging scheme. Such \fixme{the same pieces are connected, or similar ones are?} pieces should also be connected to the modules covering the opposite drift region, when in final position.
\item An additional 
set of small panels should be installed on the outer modules of the FC to extend the GP over the vertical FC walls, which will  
further constrain the electric field in these regions. A FEA 
shows that the optimized overhang distance is 20~cm, provided LAr is at 40~mm above the bottom of the GP. The maximal residual field in this configuration is of the order of 13~kV/cm, with less than 1~kV/cm field in the gas phase.
\end{itemize}

%%%%%%%%%%%%%%%%%%%%%%%%%  -- these sections are not needed. I moved the little text there was up, along with the figures. Anne.

%\subsection{Designs of the Field Cage Modules}

%%%%%%%%%%%%
%\subsubsection{Top and bottom modules}

%The main structure of the top/bottom field cage, constructed of two main I-beams that are 3.6-m long, and four cross I-beams that connect main I-beams for structural stability.

%%%%%%%%%%%%
%\subsubsection{End wall modules}


%%%%%%%%%%%%
\subsection{Interfaces to other TPC components}

\subsubsection{FC to CPA}

On the top and bottom of the TPC, hinges connect each field cage module to two CPA columns.  This design allows the field cage modules to be pre-attached to the CPAs during installation, and prevents accidental damage to the APA wire plane when raising the field cage module to connect 
to the APA.

The end-wall field cage modules are hung from the CPA and APA support rails.  They do not have strong mechanical coupling to the CPAs and APAs, however, at least four resistive divider chains must be connected to the CPA's HV bus.

\begin{cdrfigure}[CPA to field cage connection]{cpa-fc-connection}{A top field cage module connected to two CPA modules}
\includegraphics[width=0.8\linewidth]{tpc_fc_cpa_connection.png}
\end{cdrfigure}


%%%%%%%%%%%%
\subsubsection{FC to APA}

The I-beams of the top/bottom field cage modules are designed to be latched onto the mating brackets on the APAs.  The design details are currently being developed.   
In addition to the mechanical connection, the ground side of the divider chain must be connected to the APA's frame ground. 


%%%%%%%%%%%%
\subsubsection{FC to beam plug}
\label{subsec:fc-beamplug}
The design of the beam plug is described in Section~\ref{sec:beamplug}. In this section we describe the interface between the beam plug and the field cage. The beam plug is installed between the field cage and primary membrane where the charged particle beam enters the cryostat. Its main function is to displace about 45 cm of passive LAr layer in that region to allow the particle beam to enter the active TPC region with minimal upstream material interactions. The beam plug is mounted onto one of the field cage support structures as shown in Figures~\ref{fig:beamplug-fc} and~\ref{fig:beamplug-fc2}. The support structure is designed with sufficient  strength and stiffness to support the weight of the beam plug.
\begin{cdrfigure}[Beam plug to CPA connection]{beamplug-fc}{Beam plug to field cage interface.}
\includegraphics[width=0.75\linewidth]{beamplug-fc.pdf}
\end{cdrfigure}
\begin{cdrfigure}[Beam plug to CPA connection]{beamplug-fc2}{Cutaway sideview of the beam plug to field cage interface.}
\includegraphics[width=0.5\linewidth]{beamplug-fc2.pdf}
\end{cdrfigure}






