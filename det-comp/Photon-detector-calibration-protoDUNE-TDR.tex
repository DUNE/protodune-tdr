
\subsection{Photon Detector UV-Light Monitoring System}
\label{sec_pd_calib}

%Items relevant to the PDS calibration are the fast and slow components of the light, photon propagation including scattering and reflections, impact of N$_2$, E-field strength, 
%and the energy range of interest. A calibration system that addresses these issues has to be both comprehensive and cost-effective, and has to be tied to the overall 
%calibration system for ProtoDUNE-SP, which includes both charge and scintillation light calibration techniques. 
%In addition, there is a need to evaluate relative efficiencies of multiple PD units and monitor response and stability of the system as a function of time.

A UV-light-based monitoring system that will serve to monitor the relative performance and time resolution of the system has been designed. % and is described here.
The system will consist of a set of UV LEDs as light sources in the VUV wavelength range, coupled to quartz fibers, to transmit light from outside the detector volume to desired locations at the CPA within the TPC.
Light diffusers located at the CPA surface will uniformly illuminate the APA area with photon-detector system (PDS) elements.
The light sources located and fired externally, with fibers running into the cryostat to diffusers that will emit light from the CPA to the APA. 
For the ProtoDUNE-SP cryostat at the surface at CERN, the UV light system will be complementary to cosmic ray muon 
tracks and Michel electrons as means of calibration. In terms of light sources the measurements will be performed with an UV (245-280) light source.
The UV light essentially mimics physics, although at a different wavelength starting from the wavelength-shifter conversion, 
light guide propagation, photo-sensor detection and the front-end electronics readout.
	
The external UV-light monitoring system is designed with the following goals:
				
\begin{itemize}
\item Simple to implement (no active components within PD/APA, such as LEDs or fibers mounted within APA).
\item Uniformly illuminates APA surface with the light diffused from CPA locations.
\item Has a potential to be adapted for deployment in a large Far Detector in the future
\end{itemize}

In terms of technical requirements the system needs to:
\begin{itemize}
\item provide light levels down to a single p.e. at individual photon-detector channels,
\item provide higher light levels to test linearity of the PDS,
\item provide variable pulse width to test the time resolution of the photon detector response, and
\item uniformly illuminate the APA area of the detector for relative monitoring of the PDS channels
\end{itemize}

%Leon stopped fixing here 8/31

Figure~\ref{fig:fig-c-1} illustrates the system design schematically. The system consists of a 1U rack mount Light Calibration Module (LCM) sitting outside the cryostat. The LCM generates light pulses that propagate through a quartz fiber-optic cable to diffusers at the CPA to distribute the light uniformly across the photon detectors mounted within the APA.  ProtoDUNE-SP will have five light 
diffusers on the CPA plane: one in the center and four diffusers close to the CPA corners, as illustrated in Figure~\ref{fig:fig-c-2}. 
%
 \begin{cdrfigure}[UV-light monitoring system]{fig:fig-c-1}{Concept of the UV-light monitoring system for the photon detector in liquid argon.}
\includegraphics[angle=0,width=10cm,height=7cm]{calPD_concept-from-35ton.pdf}
\end{cdrfigure}
%
%
\begin{cdrfigure}[Deployment of PDS UV-light monitoring system]{fig-c-2}{The diffuse light is emitted from diffusers (top and bottom left figure) mounted at five CPA locations, indicated by arrows (right figure). 
The UV light from the light Calibration Module to diffusers is transported through quartz fiber.}
\includegraphics[angle=0,width=0.4\textwidth]{calPD_Calibration_diffuser_system_protoDUNE_slide_left_half.pdf}
\includegraphics[angle=0,width=0.55\textwidth]{calPD_Calib_diffuser_locations_protoDUNE.pdf}
\end{cdrfigure}

\fixme{left hand figure's text is way too small}
%
%
For the 280-nm light a simulation of the designed diffuse light monitoring system has been performed using TracePro, a generalized 3D light ray-tracing program with the ability 
to include bulk optical properties such as absorption, fluorescence, and birefringence in addition to surface properties such as scattering and reflection. 
As an example, Figure~\ref{fig:fig-c-4} shows simulated light distributions at an APA surface for the cases of the VUV light emitted by either the central diffuser only (left figure), 
or by the outer four diffusers simultaneously (right figure). 

%
\begin{cdrfigure}[UV-light monitoring illumination calculation]{fig:fig-c-4}{Simulated light distributions of at the APA location for the cases of the VUV light emitted by either the central diffuser only (left figure), or by outer four diffusers simultaneously (right figure).
The simulation estimate has been obtained for 35-ton detector and scaled by to 3.6 m CPA - APA distance at ProtoDUNE-SP.}
\includegraphics[angle=0,width=6.5cm,height=6cm]{calPD_4mx4m-area-3point6m-away-quantified.png}
\includegraphics[angle=0,width=6.5cm,height=6cm]{calPD_figR.pdf}
\end{cdrfigure}
%

The LCM as well as the full prototype of the light monitoring system system has been built, tested and successfully operated with 35-ton LArTPC at Fermilab. The LCM shown in Figure~\ref{fig:fig-c-3}
utilizes the logic and timing control of the photon-detector readout electronics ("SSP") unit.  
An SSP board was repackaged into a deeper rack mount chassis that accommodates a new internal 
LED Pulser Module (LPM) and an additional bulk power supply. The LPM utilizes five digital outputs from the SSP board to control the LPM pulse and its duration.  
These outputs are derived from the charge injection control logic within the SSP's FPGA.  
The even-channel SiPM bias Digital to Analog Converters (DACs)
are used to control the LPM pulse amplitude.  
The adjacent odd channels are used to read out a reference photodiode used for pulse-by-pulse monitoring of the LED light output.  
The output of the monitoring diode may be used to normalize 
the response of the SiPMs in the detector to the monitoring pulse.

%
 \begin{cdrfigure}[]{fig:fig-c-3}{Photon detector light calibration module (LCM) front (left) and back (right) ends.}
\includegraphics[angle=0,width=7cm,height=4.5cm]{calPD_LCM_front.png}\includegraphics[angle=0,width=7cm,height=4.5cm]{calPD_LCM_back.png}
\end{cdrfigure}


The data from the 35-ton test is currently being analyzed. %As en example of the system tests in former 35-ton detector in 
Figure~\ref{fig:fig-wf} shows SSP waveforms collected as a response to the light  monitoring system % described here.
%This data has been collected 
in liquid argon, with the TPC powered-on, and at the nominal value of drift high-voltage.  
%at 35-ton detector. 
The length of optical fiber cables used %with the 35-ton detector 
was $\sim$22 m total, including external fibers (7 m), and fiber feed-through and internal fibers (15 m).

%
 \begin{cdrfigure}[Detected UV-light monitoring system light pulses]{fig:fig-wf}{Photon detector light pulses detected by readout electronics. Recorded waveforms have range from few photo-electrons (left figure), to tens of photo-electrons (middle figure), to hundreds of photo-electrons (right figure).}
\includegraphics[angle=0,width=5cm,height=4cm]{calPD_example_ch21_run18573.pdf}\includegraphics[angle=0,width=5cm,height=4cm]{calPD_example_ch21_run18574.pdf}\includegraphics[angle=0,width=5cm,height=4cm]{calPD_example_ch21_run18572.pdf}
\end{cdrfigure}
%
In the DUNE prototypes (e.g., the ProtoDUNE-SP) and in future DUNE far detector modules it will be important to 
verify the proper function of photon-detector components
at various stages of the detector operation. 
Periodic light-source deployments will monitor the system's stability as a function of time. A change in relative difference of UV light responses would point towards potential wavelength shifter instability, 
or changes in SiPM gain and/or collection efficiencies. %Much of the same monitoring is expected to be doable with cosmic rays in the ProtoDUNE-SP (at surface), with 
Periodic LED monitoring runs will be complemented 
with cosmic-ray data tracked by an external hodoscope. 
%With the ProtoDUNE-SP detector one could use a well-defined muon trajectory defined by the hodoscope geometry and monitor the number of PEs per MeV of deposited charge. 
A well-defined muon trajectory defined by the hodoscope geometry could be used to monitor the number of p.e.'s per MeV of deposited charge, which in turn 
%The number of p.e.'s per photon-detector channel from the well defined muon track 
could be used as a calibration constant. This technique will likely be unusable 
%However, 
for the deep underground DUNE detectors since the cosmic ray flux may be inadequate. % for timely monitoring of the photon detectors.
	Two sets of monitoring runs are planned for the ProtoDUNE-SP detector: 
\begin{enumerate}
\item Monitoring runs with four outer diffusers run simultaneously, in order to
   \begin{itemize}
   \item measure response of PDS channels in multi-p.e. range and get integrated number of event samples for each channel (for maximum light output);
   \item test  the dynamic range from 1 p.e. to maximum number of p.e.'s; and
   \item repeat runs periodically to trace any changes in channel response.
    \end{itemize}
\item Runs with central diffuser only, in order to
   \begin{itemize}
   \item perform initial test runs that will reveal malfunctioning channels, if any;
   \item perform timing measurements with the 10--50-ns pulses; and
    \item verify time resolution of the PDS.
   \end{itemize}
\end{enumerate}

\fixme{How long do these monitoring run take ? When would they be taken (relative to beam and comics running) ? Some interference with light from comics is expected; how will this be addressed ?}

The controlled source of light %described here 
in this monitoring system will be used to perform time offset and time resolution measurements.  
Many effects contribute to a finite time resolution, relative time offset of photon-detector channels, scintillation time constants, 
photon conversion with wavelength shifter, photon propagation through photon-detector paddle, SiPM jitter, and FEE resolution. 
Most of these effects are constant and can be individually 
measured on the bench.  The UV light monitoring system will monitor overall stability of the photon detector in both time
and amplitude.
