%\chapter{det-comp}


%%%%%%%%%%%%%%%%%%%%%%%%%%%%%%%%%%%%%%%%%%%%%%
%\section{Anode Plane Assemblies}

%%%%%%%%%%%%%%%%%%%%%%%%%%%%%%%%%%%%%%%%%%%%%%
%\section{Cathode Plane Assemblies}

%%%%%%%%%%%%%%%%%%%%%%%%%%%%%%%%%%%%%%%%%%%%%%
%\section{Field Cage}

%%%%%%%%%%%%%%%%%%%%%%%%%%%%%%%%%%%%%%%%%%%%%%
%\section{HV components}

%%%%%%%%%%%%%%%%%%%%%%%%%%%%%%%%%%%%%%%%%%%%%%
\section{Photon detection system}

Intro  Exists?  Retrieve from Norm and update as needed --Leon
The scope of the photon detector (PD) system for the DUNE far detector
reference design includes design, procurement, fabrication,
testing, delivery and installation of the following components:
\begin{itemize}
\item light collection system including wavelength shifter and light guides,
\item silicon photo-multipliers (SiPMs),
\item readout electronics,
\item calibration system, and
\item related infrastructure (frames, mounting boards, etc.).
\end{itemize}

LAr is an excellent scintillating medium and the photon detection
system will exploit this property in the far detector.  With an
average energy of 19.5~eV needed to produce a photon (at zero field),
a typical particle depositing 1~MeV in LAr will generate
40,000~photons with wavelength of 128~nm. At higher fields this will
be reduced, but at 500~V/cm the yield is still $\sim$20,000~photons
per MeV. Roughly 1/4 of the photons are promptly emitted with a
lifetime of about 6~ns while the rest have a lifetime of
1100--1600~ns. Prompt and delayed photons are detected in
  precisely the same way by the photon detection system. LAr is
highly transparent to the 128-nm VUV photons with a Rayleigh
scattering length of (66~$\pm$~3)~cm~\cite{Rayleigh} and absorption
length of $>$200~cm; this attenuation length requires a LN2
  content of less than 20~ppm. The relatively large light yield makes
the scintillation process an excellent candidate for determination of
$t_0$ for non-beam related events. Detection of the scintillation
light may also be helpful in background rejection and triggering on
non-beam events.

The photon detection system reference design described in this section
meets the required performance for light collection for the DUNE far
detector. This includes detection of light from proton decay
candidates (as well as beam neutrino events) with high efficiency to
enable 3D spatial localization of candidate events. The TPC will
provide supernova neutrino detection. 
The photon system will provide the $t_0$ timing of
events relative to TPC timing with a resolution better than 1~$\mu$s
(providing position resolution along drift direction of a couple of mm). 


%%%%%%%%%%%%%%%%%%%%%%%%%%
\subsection{Light Guides and Radiators}

%%%%%%%%%%%%%
\subsubsection{Rel Light yield of alternatives}
      -IU Tallbo -- Denver/Stuart
     Results of testing
     Fit to Determine Light Yield
     Fit/Ratio vs. Distance to determine atten length     

%%%%%%%%%%%%%
\subsubsection{ Absolute light yield}
     Connection to Absolute -- Tallbo-35T  -- Denver Tallbo, Celio,Jonathan,Alex --35T
     (This could get dropped, or altered depending on status of 35T)
     Light yield by paddle
     Comparison to Simulation
     pe's / MeV 

%%%%%%%%%%%%%%%%%%%%%%%%%%
\subsection{Radiator/guide lifetime}
     IU Tallbo and SDSM and T  -- IU design SDSM and T testing  CSU QA/QC
     Design/Mfr Radiator and guide
     Design/Mfr dipped guide -- Toups
     Testing - cryogenic cycling 
          rad
          guide
          dipped guide
     QA/QC tracking and testing plan  

%%%%%%%%%%%%%%%%%%%%%%%%%%
\subsection{Sensors}

%%%%%%%%%%%%%
\subsubsection{SiPM performance and testing}
      -- NIU? Caltech?
     Testing plan for assembled SiPM boards
     QA/QC tracking for boards     

%%%%%%%%%%%%%
\subsubsection{SiPM Lifetime}
      -- IU HU Caltech NIU CSU 
     Long term testing -- IU light/dark tests     
     (maybe not needed for protoDUNE TDR, but will for CD-3)

%%%%%%%%%%%%%%%%%%%%%%%%%%
\subsection{Packaging/mounting design}
      -- CSU  -- Dave
     Mechanical Design Mounting 
     Mechanical Design Cabling
     Mechanical Design SiPM mounting board
     QA/QC tracking for components


%%%%%%%%%%%%%%%%%%%%%%%%%
\subsection{QC Procedures}



