%\chapter{det-comp}


%%%%%%%%%%%%%%%%%%%%%%%%%%%%%%%%%%%%%%%%%%%%%%
%\section{Anode Plane Assemblies}

%%%%%%%%%%%%%%%%%%%%%%%%%%%%%%%%%%%%%%%%%%%%%%
%\section{Cathode Plane Assemblies}

%%%%%%%%%%%%%%%%%%%%%%%%%%%%%%%%%%%%%%%%%%%%%%
%\section{Field Cage}

%%%%%%%%%%%%%%%%%%%%%%%%%%%%%%%%%%%%%%%%%%%%%%
\section{TPC high-voltage (HV) components}

%%%%%%%%%%%%%%%%%%%%%%%%
\subsection{Scope and requirements}

The TPC high voltage (HV) components include the HV power supply, cables,
filter circuit, feedthrough, attachment to the resistive cathode plane
arrays, the HV bus providing low-resistance connections between CPAs,
connections to the field cage, and devices for monitoring steady state
and transient conditions of current and voltage.

A schematic of the complete TPC HV circuit is shown in Figure~\ref{fig:TPCHVcircuit}.

\begin{cdrfigure}[TPC HV circuit]{TPCHVcircuit}{A schematic of the TPC high voltage circuit.}
  \includegraphics[width=0.95\textwidth]{VR_TPC_HV_schem-mod-1}
\end{cdrfigure}


The cathode plane will be biased at \SI{-180}{kV} to provide the
required \SI{500}{V/cm} drift field.  It will be
powered by a dedicated HV power supply through an RC filter and
feedthrough.  The power supply for the cathode plane must be able
to provide \SI{-200}{kV}.  The output voltage
ripple must not introduce more than 10\%\fixme{(?) check ripple requirement}
of the equivalent thermal
noise from the front-end electronics. The power supply must be
programmable to shut down its output at a certain current
limit. During power on and off, including output loss (for any
reason), the voltage ramp rate at the feedthrough must be controllable
to prevent damage to the in-vessel electronics from excess charge
injection. The high-voltage feedthrough must be able to withstand \SI{-250}{kV}
at their center conductors in a \SI{1}{atm} argon gas environment when
terminated in liquid argon.

\fixme{The above is excellent, IMHO, and should serve as a model intro.  (Anne)}
%% input from F. Pietropaolo

%%%%%%%%%%%%%%%%%%%%%%%%
\subsection{HV feedthrough design, power supply and cabling}
%In the present baseline option, 
In the design of the HV feedthrough for ProtoDUNE-SP, the procurement of the power supply and HV cables and possibly the HV filtering scheme, will take advantage of the strong synergies between the single phase and dual phase prototypes. In particular:

\begin{itemize}	
\item The Heinzinger 300-kV power supply (residual ripple less than $10^{-5}$) and the related HV cable foreseen for the DP detector are also well suited for the SP, although used at lower voltage.
\item The present DP HV feedthrough design is easily adapted to the SP without any major modification in the dimensions or in the mechanical features.
\item The filtering scheme and the monitoring system is probably more demanding on the SP detector, due to the more sensitive front-end electronics, however a common development with the DP could be advantageous, allowing to get the same HV distribution chain for both the SP and the DP protoDUNE detectors.
\item Common spare components are also planned. %envisaged.
\end{itemize}

The %present 
design of the 300-kV feedthrough is based on the very successful construction technique adopted for the ICARUS HV feedthrough, which was operated at 75~kV without interruption for more than three years without any failure. The feedthrough was also successfully operated for several days at 150~kV.  \fixme{In a context different than icarus?}
%A coaxial geometry is adopted: 
The design is based on a coaxial geometry, with an inner conductor (HV) and an outer conductor (ground) insulated by UHMW PE  as shown in Figure~\ref{fig:hv-feed-through}. The outer conductor, made of a stainless-steel tube, surrounds the insulator, extending inside the cryostat up to the LAr level. %By such a 
In this geometry the electric field is %always 
confined in regions occupied by high-dielectric-strength media (UHMW PE and LAr). \fixme{Need to spell out?} The inner conductor is made of a thin-walled stainless steel tube to minimize the heat input and to avoid the creation of argon gas bubbles around the HV lower end. A contact, welded at the upper end for the
connection to the HV cable and a round-shaped elastic contact for the connection to the cathode, screwed at the lower end, completes the inner electrode. Special care has been taken in the assembly to ensure complete filling with the PE dielectric of the space between the inner and outer conductors, and to guarantee leak-tightness at ultra-high-vacuum levels.

The design of the full HV chain planned %foreseen 
for the DP detector will be finalized  after a series of tests on a prototype feedthrough and on the Heinzinger 300-kV Power Supply, which are presently ongoing at ETHZ and CERN. An alternative but similar design for the HV feedthrough is also under development at UCLA. A final decision on the design option will be based on the maximum achievable HV, reliability and stability at the design HV, and residual noise performance.
\fixme{Do we want prev pgraph?}


\begin{cdrfigure}[HV feedthrough]{hv-feed-through}{Preliminary design of the DP HV feedthrough.}
\includegraphics[width=0.95\textwidth]{tpc_HVFT300kV-1.png}
\end{cdrfigure}
\fixme{Figure~\ref{fig:hv-feed-through} could use a bit more explanation in the caption; the text of the figure is hard to read}
%%


Despite the ripple specification of $10^{-5}$ from the Heinzinger power supply, the ripple amplitude is still too large for the TPC.  At 2.5-m drift distance (the short drift configuration), the capacitive coupling between the cathode and the grid plane, assuming a simple parallel plate capacitor, is about 73~pF.  About 20\% of this coupling goes to the first induction plane (U).  There are 800 U wires per APA, and each wire has half its length facing the CPA, yielding a  capacitance between a U wire and the CPA of about 18~fF.  To inject 100e noise into a U channel, it only needs about 0.9~mV of ripple on the cathode, while the power supply at 180~kV will generate ripple voltage of 1.8~V. \fixme{the 1.8 is unwanted? pls clarify} Obviously for the SP TPC, further filtering of the HV output with attenuation factor of $>2000$ is needed. 

%The current candidate for the high-voltage power supplies is the 
%Heinzinger PNChp series, which has the lowest output ripple 
%specification.
Additional filtering of the voltage ripples is done
through the intrinsic HV cable capacitance and series resistors
installed inside the filter box. Established techniques and practices
will be implemented to eliminate micro-discharges and minimize
unwanted energy transfer in case of a HV breakdown.

%We have two current candidates for the feedthrough: a feedthrough
%designed by and under construction at UCLA, and the dual-phase
%feedthrough design.

%%%%%%%%%%%%%%%%%%%%%%%%
\subsection{HV bus}

As described in Section~\ref{sec:cpa}, the cathode planes will be
resistive, with electrical connections at the corners, in order to
control the energy delivered in any discharges.  A low-resistance
``high voltage bus'' will provide the high voltage to the field cage
circuit and cathodes with a voltage drop much less than 0.1\% of the
cathode voltage. The location of the bus with respect to the CPA frame is shown in Figure~\ref{fig:HVbusmodel}. Field-shaping electrodes on the faces of the CPA
frames will be part of the field cage circuit, described in Section~\ref{detcompsec-fc}. %the field cage section above. 
Field cage electrodes on the outer edges of the
CPA frames will be held at the cathode potential to provide field
uniformity and to protect the HV bus from discharge.  The feedthrough
will connect to a high voltage cup on one side of a CPA at one end of
the cathode plane.  Interconnection of the bus between CPAs will be made
through HV cables passed through the CPA frames.  See
Figure~\ref{fig:HVbus} for drawings of the high voltage bus and its
interfaces.

\begin{cdrfigure}[Model of the HV bus]{HVbusmodel}{A perspective view of CPA frame showing the location of the HV bus cable and attachments to the HV cup and resistive cathode, with CPA frame electrodes omitted to make HV bus visible.}
\includegraphics[height=0.35\textheight]{DUNE_SP_CPA_Design_Update-slide17-mod}
\end{cdrfigure}

\begin{cdrfigure}[HV bus connections and equipotential contour diagram]{HVbus}{Top: A sketch showing interconnection between two CPAs. Bottom: A transverse cross-section with equipotential lines around the HV bus and CPA frame.}
\includegraphics[height=0.3\textheight]{HVbus-connections}
\includegraphics[height=0.3\textheight]{HVbus-vcontour}
\end{cdrfigure}

\fixme{Do the three images in Figure~\ref{fig:HVbus} need to be together? The bottom two would be more readable if they were separate and could be bigger}

%%%%%%%%%%%%%%%%%%%%%%%%
\subsection{HV monitoring}

HV circuit monitoring devices include a toroid transformer to detect
spikes and noise in the current draw, and a monitoring point at the end
of the field cage resistor chain, which also provides a means to
control field-shaping around the edge of the
APA.

%%%%%%%%%%%%%%%%%%%%%%%%
\subsection{HV component testing}

To ensure safe and reliable operation, the HV components will be
tested at a much higher voltage than expected in routine operation
($\sim\SI{250}{kV}$) in LAr. Among these tests will be a planned
``full scale'' high voltage test at Fermilab in which all components
are subjected to the full voltage and field in liquid argon in the
35-ton cryostat. \fixme{add 35-t reference here}
