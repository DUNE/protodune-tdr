%\chapter{det-comp}


%%%%%%%%%%%%%%%%%%%%%%%%%%%%%%%%%%%%%%%%%%%%%%
%\section{Anode Plane Assemblies}

%%%%%%%%%%%%%%%%%%%%%%%%%%%%%%%%%%%%%%%%%%%%%%
%\section{Cathode Plane Assemblies}

%%%%%%%%%%%%%%%%%%%%%%%%%%%%%%%%%%%%%%%%%%%%%%
%\section{Field Cage}

%%%%%%%%%%%%%%%%%%%%%%%%%%%%%%%%%%%%%%%%%%%%%%
\section{HV components}

The TPC high voltage components include the HV power supply, cables,
filter circuit, feedthrough, attachment to the resistive cathode plane
arrays, the HV bus providing low-resistance connections between CPAs,
connections to the field cage, and devices for monitoring steady state
and transient conditions of current and voltage.

A schematic of the complete TPC HV circuit is shown in Fig.\ \ref{fig:TPCHVcircuit}.

\begin{cdrfigure}[TPC HV circuit]{TPCHVcircuit}{A schematic of the TPC high voltage circuit.}
  \fixme{put in the HV schematic}
  %\includegraphics[width=0.8\textwidth]{TPCHVcircuit-Figure}
\end{cdrfigure}


The two cathode planes are biased at \SI{-180}{kV} to provide the
required \SI{500}{V/cm} drift field.  Each cathode plane will be
powered by a dedicated HV power supply through an RC filter and
feedthrough.  The power supplies for the cathode planes must be able
to provide \SI{-200}{kV} at \SI{1}{mA} current.  The output voltage
ripple must not introduce more than 10\% of the equivalent thermal
noise from the front-end electronics. The power supplies must be
programmable to shut down their output at a certain current
limit. During power on and off, including output loss (for any
reason), the voltage ramp rate at the feedthrough must be controllable
to prevent damage to the in-vessel electronics from excess charge
injection. High-voltage feedthroughs must be able to withstand \SI{-250}{kV}
at their center conductors in a \SI{1}{atm} argon gas environment when
terminated in liquid argon.

The current candidate for the high-voltage power supplies is the
Heinzinger PNChp series, which has the lowest output ripple
specification. Additional filtering of the voltage ripples is done
through the intrinsic HV cable capacitance and series resistors
installed inside the filter box. Established techniques and practices
will be implemented to eliminate micro-discharges and minimize
unwanted energy transfer in case of an HV breakdown.

We have two current candidates for the feedthrough: a feedthrough
designed by and under construction at UCLA, and the dual-phase
feedthrough design.

As described in the CPA section above, the cathode planes will be
resistive, with electrical connections at the corners, in order to
control the energy delivered in any discharges.  A low resistance
``high voltage bus'' will provide the high voltage to the field cage
circuit and cathodes with voltage drop much less than 0.1\% of the
cathode voltage.  Field-shaping electrodes on the faces of the CPA
frames will be part of the field cage circuit, described in the field
cage section above. Field cage electrodes on the outer edges of the
CPA frames will be held at the cathode potential to provide field
uniformity and to protect the HV bus from discharge.  The feedthrough
will connect to a high voltage cup on one side of a CPA at one end of
the cathode plane.  Interconnection of the bus between CPAs will be
through HV cables passed through the CPA frames.  See
Fig.~\ref{fig:HVbus} for drawings of the high voltage bus and its
interfaces.

\begin{cdrfigure}[The HV bus.]{HVbus}{(a) Transverse cross-section of CPA top frame showing the location of the HV bus cable and equipotential countours; (b) front view showing location of cable and interconnect between two CPAs; (c) Attachment to HV cup on outer CPA frame.}
  \fixme{put in the nice figures of the HV bus}
  %\includegraphics[width=0.8\textwidth]{HVbus-figure}
\end{cdrfigure}

HV circuit monitoring devices include a toroid transformer to detect
spikes and noise in the current draw and a monitoring point at the end
of the field cage resistor chain, which also provides a means to
control field shaping around the edge of the
APA. (Fig.\ \ref{fig:TPCHVcircuit}.)

To ensure safe and reliable operation, the HV components will be
tested at a much higher voltage than expected in routine operation
($\sim\SI{250}{kV}$) in LAr. Among these tests will be a planned
``full scale'' high voltage test at Fermilab in which all components
are subjected to the full voltage and field in liquid argon in the
35-ton cryostat.


