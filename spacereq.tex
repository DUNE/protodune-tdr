%\chapter{spacereq}

%%%%%%%%%%%%%%%%%%%%%%%%%%%%%%%%%%%%%%%%%%%%%%
\section{Installation space and clean room}

Figure~\ref{fig:cryostat-in-ehn1} shows the ProtoDUNE-SP cryostat in EHN1.  The cryostat will be constructed in a pit inside the building and will be surrounded on three sides by the pit walls.  On the open side of the cryostat we will construct an ISO 8 clean room from which to construct, test and install the TPC.  In addition to air purity, the lighting inside the clean room and any temporary lighting inside the cryostat will be filtered to limit the exposure of the PDs to UV light.  This filtering limits UV less than 450 nm from the ambient light.  

This is also the face of the cryostat that will have the temporary construction opening (TCO) that will allow us to insert and install the large TPC components.  Outside of the clean room is a material pass through or material SAS that will have a removable roof in which to lower the TPC elements into the clean environment from the gallery floor.

\begin{cdrfigure}[ProtoDUNE SP cryostat in EHN1]{cryostat-in-ehn1}{ProtoDUNE SP cryostat in EHN1}
\includegraphics[width=0.8\textwidth]{sp-cryostat-in-ehn1}
\end{cdrfigure}

A naming convention has been established to label the four sides of the cryostat for logistical purposes.  The plan view in Figure~\ref{fig:cryo-side-names} shows this naming convention.  The upper side is Jura, the lower is Saleve, the left is beam and right is downstream.

\begin{cdrfigure}[Conventions for labeling the four sides of the cryostat]{cryo-side-names}{Conventions for labeling the four sides of the cryostat}
\includegraphics[width=0.8\textwidth]{naming-conv-cryo-sides}
\end{cdrfigure}

Figure~\ref{fig:elev-view-cryostat} is an elevation section view of the cryostat.  For reference is shows the position of the TCO and the location of the integrated cold testing stand that will be described in the next section.  The TCO will be used to bring personnel, materials, access equipment and TPC components inside the cryostat. 

\begin{cdrfigure}[Elevation section view of the cryostat]{elev-view-cryostat}{Elevation section view of the cryostat}
\includegraphics[width=0.8\textwidth]{elev-view-cryostat}
\end{cdrfigure}


Inside the clean room, there will be a series of rails to facilitate the movement of the TPC components during the test and installation process.  The conceptual layout of these rails are shown in Figure~\ref{fig:rails-in-cleanroom}.  These rails will be positioned vertically at the same height of the DSS rails inside the cryostat.  Through the TCO, a temporary rail will be installed that bridges the DSS and the rails in the clean room.  All of the large components of the cryogenic piping and TPC will be supported from these rails on movable trolleys to move them inside the cryostat.  

\begin{cdrfigure}[Conceptual layout of rails in cleanroom]{rails-in-cleanroom}{Conceptual layout of rails in cleanroom to facilitate movement of TPC components}
\includegraphics[width=0.8\textwidth]{rails-in-cleanroom}
\end{cdrfigure}


Figure~\ref{fig:rails-in-cleanroom} also shows the approximate dimensions for the material SAS and the footprint of the clean room space.  This space is limited by the pit walls on two sides and the supports for the beam and beam instrumentation on the other.  

Figure~\ref{fig:sas-locations} shows the planned locations for all of the activities that will be performed inside the clean room.  

\begin{cdrfigure}[Approximate dimensions for the material SAS plus cleanroom footprint]{sas-locations}{Approximate dimensions for the material SAS and the footprint of the clean room space}
\includegraphics[width=0.8\textwidth]{sas-locations}
\end{cdrfigure}


Material will pass through the SAS through a large set of doors.  These doors will be closed while material is lowered into the SAS.  Once the roof of the SAS is closed, these doors will be opened to move the material into the clean room.  This is done to reduce the contamination of the clean room space from the ambient conditions of EHN1.  

The activities planned are as follows:
\begin{itemize}
\item Assembly of the CPA panels into CPA modules.  This requires a large flat surface approximately 1500 mm x 6500 mm to join the three CPA panels into one CPA module.  Once these are assembled, an overhead hoist located near the TCO will be used to translate the CPA module from horizontal to vertical and place it for attachment to the clean room rails.  
\item Attachment of the upper and lower FC assemblies to the CPA modules.  Once two of the CPA modules are attached to the clean room rail, two upper and two lower FC assemblies will be attached.  This will be done with the same overhead hoist located near the TCO.
\item Installing the PDs in the APA frames.  Each PD module will be unpacked, tested and then installed in the APA frame. 
\item Mounting of the CE onto the APAs.  Each of the CE modules with cables will be unpacked, tested and then mounted onto the APA frame.  
\item Integrated testing of APA with PD and CE.  
\end{itemize}

%%%%%%%%%%%%%%%%%%%%%%%%%%%%%%%%%%%%%%%%%%%%%%
%\section{QA/QC and testing space}
\section{Testing}

All of the testing for the TPC components will be done inside the clean room or inside the cryostat.  This will be described for the major sub system components below.
Once the APAs are moved into the clean room, a connectivity test will be performed to look for shorted or broken wires.  It will also be inspected visually for broken wires.  Tension tests will be performed on some fractions of the wires and the results will be compared to the ones taken at the production facility.

For the PDs, each PD will be unpacked and placed inside a full length dark box with a scanning VUV light source. This will be an identical test set up to the one used at the production facility.  Each test will take approximately 2 hours, which is the planned time for the installation and cabling of a PD into the APA.  

The CE modules will be delivered as an assembly.  The CE enclosure will have the electronics boards mounted inside and the internal cable from the harness connected and tested.  Each module will undergo an acceptance test when they are unpacked inside the clean room.  Once tested, they will be installed on the top of the APA frame from either a scaffold or a scissor lift located in the clean room.  A connectivity test will be performed on each module to ensure the electrical connections.  

After the APA has been fully integrated with the PD and CE, it will be moved via the rails in the cleanroom to the integrated cold testing stand.  This test stand, shown in Figure \fixme{6}, is a large insulated box that is light tight for PD testing and a Faraday shield for CE testing.  The test stand is shown with an APA inside and the end cover removed.  
%
\begin{cdrfigure}[short caption for table of figures]{label}{6 long caption that appears below picture}
%  \includegraphics[width=0.8\textwidth]{protoDUNE_cryostat.png}
\end{cdrfigure}
%
At the top of the box, there will be a crossing tube, similar to those in the cryostat, with a conflat fitting that will accept the warm – cold interface flange for the PD and CE cable connections.  The PD and CE cables will be routed and connected to their flanges, the APA will be moved inside and the end cover, that completes the Faraday cage, installed so that the integrated electronics testing can begin.  A series of warm tests will be performed.  This is described in \fixme{need to find... XXsectionXX}.  After the warm tests are complete, the inner volume of the enclosure will be purged with dry gas to reduce the moisture inside.  After this gas purge, the inner volume will be slowly cooled down using nitrogen gas to a temperature of approximately 100 K.  The rate of cooldown is to be less than 10 K/hr which is the same for the cryostat.  The cooldown system is being designed to maintain the inner volume near 100 K for approximately 48 hours.  The cold testing of this can be performed during the cooldown period, during these 48 hr and during the warmup.  After the testing is complete and the box has been purged of nitrogen with room air, it will be opened, the APA removed and the cables disconnected and secured for movement on the rail system.   

Each of the CPA panels will be visually inspected for damage to the resistive coating on the panels.  At the panels are assembled into a CPA module, electrical connections will be made and tested between the panels.  As the modules are mounted on the clean room rails, the modules will be electrically connected between the modules.  This connection will be tested in the clean room and when the CPAs are moved inside the cryostat to their final position.  

The FC assemblies will be visually inspected for damage as they are removed from their shipping containers.  The electrical connections between the divider chains will be tested.  Once attached to the CPA, an electrical connection between the CPA and FC must be made to carry the HV to the FC.  This will be tested in the clean room before moving the assembly into the cryostat. 


